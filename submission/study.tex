\documentclass[11pt,]{article}
\usepackage{lmodern}
\usepackage{amssymb,amsmath}
\usepackage{ifxetex,ifluatex}
\usepackage{fixltx2e} % provides \textsubscript
\ifnum 0\ifxetex 1\fi\ifluatex 1\fi=0 % if pdftex
  \usepackage[T1]{fontenc}
  \usepackage[utf8]{inputenc}
\else % if luatex or xelatex
  \ifxetex
    \usepackage{mathspec}
    \usepackage{xltxtra,xunicode}
  \else
    \usepackage{fontspec}
  \fi
  \defaultfontfeatures{Mapping=tex-text,Scale=MatchLowercase}
  \newcommand{\euro}{€}
\fi
% use upquote if available, for straight quotes in verbatim environments
\IfFileExists{upquote.sty}{\usepackage{upquote}}{}
% use microtype if available
\IfFileExists{microtype.sty}{%
\usepackage{microtype}
\UseMicrotypeSet[protrusion]{basicmath} % disable protrusion for tt fonts
}{}
\usepackage[margin=1.0in]{geometry}
\ifxetex
  \usepackage[setpagesize=false, % page size defined by xetex
              unicode=false, % unicode breaks when used with xetex
              xetex]{hyperref}
\else
  \usepackage[unicode=true]{hyperref}
\fi
\hypersetup{breaklinks=true,
            bookmarks=true,
            pdfauthor={},
            pdftitle={Microbiota predict Clostridium difficile severity in germ-free mice colonized with human feces},
            colorlinks=true,
            citecolor=blue,
            urlcolor=blue,
            linkcolor=magenta,
            pdfborder={0 0 0}}
\urlstyle{same}  % don't use monospace font for urls
\setlength{\parindent}{0pt}
\setlength{\parskip}{6pt plus 2pt minus 1pt}
\setlength{\emergencystretch}{3em}  % prevent overfull lines
\setcounter{secnumdepth}{0}

%%% Use protect on footnotes to avoid problems with footnotes in titles
\let\rmarkdownfootnote\footnote%
\def\footnote{\protect\rmarkdownfootnote}

%%% Change title format to be more compact
\usepackage{titling}

% Create subtitle command for use in maketitle
\newcommand{\subtitle}[1]{
  \posttitle{
    \begin{center}\large#1\end{center}
    }
}

\setlength{\droptitle}{-2em}
  \title{\textbf{Microbiota predict \emph{Clostridium difficile} severity in
germ-free mice colonized with human feces}}
  \pretitle{\vspace{\droptitle}\centering\huge}
  \posttitle{\par}
  \author{}
  \preauthor{}\postauthor{}
  \date{}
  \predate{}\postdate{}

\usepackage{helvet} % Helvetica font
\renewcommand*\familydefault{\sfdefault} % Use the sans serif version of the font
\usepackage[T1]{fontenc}

\usepackage[none]{hyphenat}

\usepackage{setspace}
\doublespacing
\setlength{\parskip}{1em}

\usepackage{lineno}

\usepackage{pdfpages}


\begin{document}

\maketitle


\vspace{35mm}

Running title: Microbiota predict \emph{C. difficile} severity in
humanized mice

\vspace{35mm}

Kaitlin J. Flynn\textsuperscript{1}, Nicholas
Lesniak\textsuperscript{1}, Alyxandria M. Schubert\textsuperscript{2},
Hamide Sinani\textsuperscript{?}, and Patrick D.
Schloss\textsuperscript{1\(\dagger\)}

\vspace{40mm}

\(\dagger\) Corresponding author:
\href{mailto:pschloss@umich.edu}{\nolinkurl{pschloss@umich.edu}}

1. Department of Microbiology and Immunology, University of Michigan,
Ann Arbor, Michigan 48109

2. Food and Drug Administration?

3. Department for Hamide?

\newpage
\linenumbers

\subsubsection{Abstract}\label{abstract}

\emph{Clostridium difficile} causes diarrheal disease when it
successfully colonizes a dysbiotic gut microbial community. Current
mouse models to study \emph{C. difficile} infection (CDI) rely on
pre-treatment with antibiotics to disrupt the mouse microbiome prior to
inoculation. This model does not allow for analysis of human-associated
microbial community members that modulate \emph{C. difficile}
colonization and expansion. To study human-associated microbes in the
context of CDI, we inoculated germ-free C57BL/6 mice with one of 16
human fecal samples from diarrheal or healthy donors and challenged with
C. difficile 14 days later. Five unique donor-mice combinations resulted
in severe CDI while the remaining 11 only experienced mild disease. Both
healthy and diarrheal donors were susceptible to colonization and severe
symptoms of CDI. To determine if specific microbes were associated with
disease severity outcomes, we built a classification Random Forest
machine learning model based on relative abundance data of the
communities prior to infection. The model identified a number of
bacterial populations associated with the development of severe CDI,
including \emph{Bacilliales, Ruminococcaceae, Ruminococcus,
Staphylococcus, Streptococcus and Bacteriodetes}. Additionally, a
regression model accurately predicted colonization levels of \emph{C.
difficile} at one to ten days post-infection. This model explained 99\%
of the variance in the number of CFU isolated from mouse stool. Members
of \emph{Lachnospiraceae, Parabacteroides, Bacteroidales, Bacteroidetes,
Porphyromonadaceae} and unclassified \emph{Bacteria} families were
predictive of future \emph{C. difficile} colonization levels. Finally,
challenging these mice with different strains of \emph{C. difficile}
revealed that susceptible human-associated microbial communities were
prone to severe disease independent of strain type. Taken together these
results suggest that human-associated microbial communities can be
recapitulated in germ-free mice and used to characterize dynamics of
CDI. Because both healthy and diarrheal patients were susceptible to
severe CDI, machine-learning models are useful to identify bacterial
populations that allow colonization and contribute to the development of
\emph{C. difficile} associated disease in humans.

\newpage

\subsubsection{Introduction}\label{introduction}

Clostridium difficile is an opportunistic pathogen of the human lower
gastrointestinal tract. Disruption of the native microbial community of
the gut by antibiotics is the most common risk factor for development of
C. difficile infection (CDI) (1). C. difficile is a spore-forming
bacteria and can persist on abiotic surfaces and is not readily killed
by ethanol-based hand-sanitizers, putting hospital patients particularly
at risk. Indeed, \textasciitilde{}12\% of hospital acquired infections
in the United States are due to C. difficile and result in up to 15,000
deaths annually (2).

Murine models to study CDI typically rely on treating
conventionally-raised mice with antibiotics either in drinking water or
by injection to induce susceptibility (3, 4). This model provides a
convenient way to study C. difficile pathogenesis and virulence factors.
Numerous microbiome studies have been performed using this model to
determine the antibiotic classes (5), starting microbial community (6)
and metabolites (7) that impact development and severity of CDI. While
informative, these studies are somewhat removed from human disease
because they only examine mouse-associated microbial communities.

Gnotobiotic or germ-free mouse models have been used for a range of
studies of CDI, including assessment of species-specific interactions
between C. difficile and competing microbial community members (8),
analysis of nutrient restriction (9), in vivo transcriptomics of C.
difficile and examination of host immune response to CDI (10). Further,
CDI therapeutics such as antibiotics and fecal microbiota transplants
have been tested extensively in a gnotobiotic-piglet or
piglet-to-gnotobiotic-mouse model of disease (11), (12). Pigs have a
longer digestive tract with components more similar to humans than mice
and are typically infected by strains typical in human infection (13).
However, the murine and porcine microbiomes typically do not resemble
those of the human gut.

The power of the gnotobiotic models to study CDI has been further
realized by first inoculating germ-free mice and piglets with human
stool microbes. In one study, germ-free piglets were acutely colonized
with human feces for one week and then treated with tigecycline. After
challenge with C. difficile none of the antibiotic-treated piglets
succumbed to infection, while some of the untreated human-colonized pigs
did (11). Further, germ-free mice colonized with human feces were bred
over several generations to create a cohort of mice with identical
human-derived microbiomes (14). These mice were subsequently treated
with a five-antibiotic cocktail to induce dysbiosis and then were
successfully colonized by C. difficile (14). While informative, these
studies were limited in their use of only one human donor as input
inoculum. In order to best understand the impact of C. difficile
pathogenesis on human disease, we must have a laboratory model that
allows for study of a variety of human-derived microbiomes.

To test the impact of individual human microbiomes on CDI, we colonized
germ-free mice with 16 different human stool donors. We then
characterized human-associated microbiome response to C. difficile
challenge. Additionally, the use of machine-learning models allowed us
to build a predictive model that classified ``at-risk'' microbiomes
prior to infection with C. difficile. These findings show that
human-associated microbiomes can be at risk for CDI even in the absence
of antibiotics and that study of mice colonized with human feces
provides a range of clinical outcomes.

\subsubsection{Results}\label{results}

\subsubsection{Discussion}\label{discussion}

\subsubsection{Materials and Methods}\label{materials-and-methods}

\newpage

Insert figure legends with the first sentence in bold, for example:

\textbf{Figure 1. Number of OTUs sampled among bacterial and archaeal
16S rRNA gene sequences for different OTU definitions and level of
sequencing effort.} Rarefaction curves for different OTU definitions of
Bacteria (A) and Archaea (B). Rarefaction curves for the coarse
environments in Table 1 for Bacteria (C) and Archaea (D).

\newpage

\subsection*{References}\label{references}
\addcontentsline{toc}{subsection}{References}

\end{document}
