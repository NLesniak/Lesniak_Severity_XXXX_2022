\documentclass[11pt,]{article}
\usepackage{lmodern}
\usepackage{amssymb,amsmath}
\usepackage{ifxetex,ifluatex}
\usepackage{fixltx2e} % provides \textsubscript
\ifnum 0\ifxetex 1\fi\ifluatex 1\fi=0 % if pdftex
  \usepackage[T1]{fontenc}
  \usepackage[utf8]{inputenc}
\else % if luatex or xelatex
  \ifxetex
    \usepackage{mathspec}
    \usepackage{xltxtra,xunicode}
  \else
    \usepackage{fontspec}
  \fi
  \defaultfontfeatures{Mapping=tex-text,Scale=MatchLowercase}
  \newcommand{\euro}{€}
\fi
% use upquote if available, for straight quotes in verbatim environments
\IfFileExists{upquote.sty}{\usepackage{upquote}}{}
% use microtype if available
\IfFileExists{microtype.sty}{%
\usepackage{microtype}
\UseMicrotypeSet[protrusion]{basicmath} % disable protrusion for tt fonts
}{}
\usepackage[margin=1.0in]{geometry}
\ifxetex
  \usepackage[setpagesize=false, % page size defined by xetex
              unicode=false, % unicode breaks when used with xetex
              xetex]{hyperref}
\else
  \usepackage[unicode=true]{hyperref}
\fi
\hypersetup{breaklinks=true,
            bookmarks=true,
            pdfauthor={},
            pdftitle={Microbiota predict Clostridium difficile severity in germ-free mice colonized with human feces},
            colorlinks=true,
            citecolor=blue,
            urlcolor=blue,
            linkcolor=magenta,
            pdfborder={0 0 0}}
\urlstyle{same}  % don't use monospace font for urls
\setlength{\parindent}{0pt}
\setlength{\parskip}{6pt plus 2pt minus 1pt}
\setlength{\emergencystretch}{3em}  % prevent overfull lines
\setcounter{secnumdepth}{0}

%%% Use protect on footnotes to avoid problems with footnotes in titles
\let\rmarkdownfootnote\footnote%
\def\footnote{\protect\rmarkdownfootnote}

%%% Change title format to be more compact
\usepackage{titling}

% Create subtitle command for use in maketitle
\newcommand{\subtitle}[1]{
  \posttitle{
    \begin{center}\large#1\end{center}
    }
}

\setlength{\droptitle}{-2em}
  \title{\textbf{Microbiota predict \emph{Clostridium difficile} severity in
germ-free mice colonized with human feces}}
  \pretitle{\vspace{\droptitle}\centering\huge}
  \posttitle{\par}
  \author{}
  \preauthor{}\postauthor{}
  \date{}
  \predate{}\postdate{}

\usepackage{helvet} % Helvetica font
\renewcommand*\familydefault{\sfdefault} % Use the sans serif version of the font
\usepackage[T1]{fontenc}

\usepackage[none]{hyphenat}

\usepackage{setspace}
\doublespacing
\setlength{\parskip}{1em}

\usepackage{lineno}

\usepackage{pdfpages}


\begin{document}

\maketitle


\vspace{35mm}

Running title: Microbiota predict \emph{C. difficile} severity in
humanized mice

\vspace{35mm}

Kaitlin J. Flynn\textsuperscript{1}, Nicholas
Lesniak\textsuperscript{1}, Alyxandria M. Schubert\textsuperscript{2},
Hamide Sinani\textsuperscript{?}, and Patrick D.
Schloss\textsuperscript{1\(\dagger\)}

\vspace{40mm}

\(\dagger\) Corresponding author:
\href{mailto:pschloss@umich.edu}{\nolinkurl{pschloss@umich.edu}}

1. Department of Microbiology and Immunology, University of Michigan,
Ann Arbor, Michigan 48109

2. Food and Drug Administration?

3. Department for Hamide?

\newpage
\linenumbers

\subsubsection{Abstract}\label{abstract}

\emph{Clostridium difficile} causes diarrheal disease when it
successfully colonizes a dysbiotic gut microbial community. Current
mouse models to study \emph{C. difficile} infection (CDI) rely on
pre-treatment with antibiotics to disrupt the mouse microbiome prior to
inoculation. This model does not allow for analysis of human-associated
microbial community members that modulate \emph{C. difficile}
colonization and expansion. To study human-associated microbes in the
context of CDI, we inoculated germ-free C57BL/6 mice with one of 16
human fecal samples from diarrheal or healthy donors and challenged with
C. difficile 14 days later. Five unique donor-mice combinations resulted
in severe CDI while the remaining 11 only experienced mild disease. Both
healthy and diarrheal donors were susceptible to colonization and severe
symptoms of CDI. To determine if specific microbes were associated with
disease severity outcomes, we built a classification Random Forest
machine learning model based on relative abundance data of the
communities prior to infection. The model identified a number of
bacterial populations associated with the development of severe CDI,
including \emph{Bacilliales, Ruminococcaceae, Ruminococcus,
Staphylococcus, Streptococcus and Bacteriodetes}. Additionally, a
regression model accurately predicted colonization levels of \emph{C.
difficile} at one to ten days post-infection. This model explained 99\%
of the variance in the number of CFU isolated from mouse stool. Members
of \emph{Lachnospiraceae, Parabacteroides, Bacteroidales, Bacteroidetes,
Porphyromonadaceae} and unclassified \emph{Bacteria} families were
predictive of future \emph{C. difficile} colonization levels. Finally,
challenging these mice with different strains of \emph{C. difficile}
revealed that susceptible human-associated microbial communities were
prone to severe disease independent of strain type. Taken together these
results suggest that human-associated microbial communities can be
recapitulated in germ-free mice and used to characterize dynamics of
CDI. Because both healthy and diarrheal patients were susceptible to
severe CDI, machine-learning models are useful to identify bacterial
populations that allow colonization and contribute to the development of
\emph{C. difficile} associated disease in humans.

\newpage

\subsubsection{Introduction}\label{introduction}

Clostridium difficile is an opportunistic pathogen of the human lower
gastrointestinal tract. Disruption of the native microbial community of
the gut by antibiotics is the most common risk factor for development of
C. difficile infection (CDI) (1). C. difficile is a spore-forming
bacteria and can persist on abiotic surfaces and is not readily killed
by ethanol-based hand-sanitizers, putting hospital patients particularly
at risk. Indeed, \textasciitilde{}12\% of hospital acquired infections
in the United States are due to C. difficile and result in up to 15,000
deaths annually (2).

Murine models to study CDI typically rely on treating
conventionally-raised mice with antibiotics either in drinking water or
by injection to induce susceptibility (3, 4). This model provides a
convenient way to study C. difficile pathogenesis and virulence factors.
Numerous microbiome studies have been performed using this model to
determine the antibiotic classes (5), starting microbial community (6)
and metabolites (7) that impact development and severity of CDI. While
informative, these studies are somewhat removed from human disease
because they only examine mouse-associated microbial communities.

Gnotobiotic or germ-free mouse models have been used for a range of
studies of CDI, including assessment of species-specific interactions
between C. difficile and competing microbial community members (8),
analysis of nutrient restriction (9), in vivo transcriptomics of C.
difficile and examination of host immune response to CDI (10). Further,
CDI therapeutics such as antibiotics and fecal microbiota transplants
have been tested extensively in a gnotobiotic-piglet or
piglet-to-gnotobiotic-mouse model of disease (11), (12). Pigs have a
longer digestive tract with components more similar to humans than mice
and are typically infected by strains typical in human infection (13).
However, the murine and porcine microbiomes typically do not resemble
those of the human gut.

The power of the gnotobiotic models to study CDI has been further
realized by first inoculating germ-free mice and piglets with human
stool microbes. In one study, germ-free piglets were acutely colonized
with human feces for one week and then treated with tigecycline. After
challenge with C. difficile none of the antibiotic-treated piglets
succumbed to infection, while some of the untreated human-colonized pigs
did (11). Further, germ-free mice colonized with human feces were bred
over several generations to create a cohort of mice with identical
human-derived microbiomes (14). These mice were subsequently treated
with a five-antibiotic cocktail to induce dysbiosis and then were
successfully colonized by C. difficile (14). While informative, these
studies were limited in their use of only one human donor as input
inoculum. In order to best understand the impact of C. difficile
pathogenesis on human disease, we must have a laboratory model that
allows for study of a variety of human-derived microbiomes.

To test the impact of individual human microbiomes on CDI, we colonized
germ-free mice with 16 different human stool donors. We then
characterized human-associated microbiome response to C. difficile
challenge. Additionally, the use of machine-learning models allowed us
to build a predictive model that classified ``at-risk'' microbiomes
prior to infection with C. difficile. These findings show that
human-associated microbiomes can be at risk for CDI even in the absence
of antibiotics and that study of mice colonized with human feces
provides a range of clinical outcomes.

\subsubsection{Results}\label{results}

\textbf{Germ-free mice inoculated with human feces as model for C.
difficile infection.} To generate mice with human-derived microbiomes,
we inoculated one cage of gnotobiotic C57/BL6 mice with one of 16
different human fecal donors. Five donors were patients that had
diarrhea that was not attributable to \emph{C. difficile} infection
while 11 donors were healthy at time of donation. Stool from a patient
that was colonized with virulent \emph{C. difficile} was used as a
positive control. After inoculation with human stool, mice were allowed
to equilibrate for 14 days. Prior to infection, stool samples were taken
from each mouse to establish baseline. Then, the \emph{C. difficile}
strain isolated from the positive control patient's sample (strain 430)
was used to infect each mouse with 100 spores. Mice were monitored for
weight loss and clinical signs of disease. Fecal samples were taken to
enumerate \emph{C. difficile} CFU and for microbiome analysis every day
for up to 10 days post-infection (Fig 1A). To ensure that the donors we
selected represented a diverse array of human microbiomes, we sequenced
the 16S rRNA genes from donor fecal inocula. Ordination of the distances
between donor communities showed that the donors each had distinctly
different communities, independent of whether the sample came from a
sick or healthy person (Fig 1B). Likewise, the starting microbial
communities of the mice on day 0 were characterizing by sequencing of
fecal pellets DNA prior to infection. Ordination of all of the mouse
communities on day 0 shows that mice were similar to each other within
each cage and donor, but distinct from other donors (Fig 1C). This
result confirmed that human-associated microbes were able to colonize
gnotobiotic mice and provide distinct initial communities to test
\emph{C. difficile} dynamics.

\textbf{C. difficile infection in mice with human-derived microbiota
cause a range of outcomes.} \emph{C. difficile} colonization was
monitored by daily plating of stool pellets for \emph{C. difficile} CFU.
Nearly all of the mice were colonized to 10\textsuperscript{5} --
10\textsuperscript{7} CFU by one day post-infection and remained
colonized at that level until the end of the experiment (Fig 2A). As one
indicator of disease, mouse weights were taken each day post-infection
and weight-loss was monitored alongside clinical signs of disease. When
mice were judged to be too ill to continue they were humanely
euthanized. Overall, disease phenotypes fell into two classes. Mice that
became severely ill and lost 20\% or more of their starting body weight
within one to two days post-infection were classified as ``severe''
whereas mice that were colonized with \emph{C. difficile} but did not
show signs of disease or severe weight loss were considered to have
``mild'' disease (Fig 2A, 2B). Interestingly, \emph{C. difficile} was
able to cause severe disease in both mice that had been colonized with
healthy stool and those colonized with diarrheal stool, suggesting
susceptibility to CDI is dependent on the composition of the starting
microbiome and not associated with donor clinical metadata.

\textbf{Results to be written} 1. Microbes present in the gut prior to
infection are predictive of C. difficile CFU and severity a. Figure 3:
Random forest to predict CFU, predictive OTUs b. Figure 4: Random Forest
predicts CDI severity, predictive OTUs 2. Propensity for severe CDI is
community-dependent and strain-independent a. Figure 5: Infection of
mice with different C. difficile strains.

\subsubsection{Discussion}\label{discussion}

• Restate results, • caveats about mouse weights • No donors were
colonization resistant, discuss donor differences • Discuss prediction
methods and outcomes • Discuss potential mechanisms for interesting OTUs
• Discuss different strain results • Future work blah blah

\subsubsection{Materials and Methods}\label{materials-and-methods}

• Mice ULAM number • Donor stool ERIN IRB shit • Bacteria/plating •
Sequencing • Data analysis, code availability • Machine learning models

\subsection{Acknowledgments}\label{acknowledgments}

Lab, sequencing core, Jhansi

\newpage

\subsubsection{Figure Legends}\label{figure-legends}

\textbf{Figure 1. Germ-free mice inoculated with human feces as a model
for C. difficile infection.} A) Experimental design. Stool from 16
healthy, diarrheal and CDI patients were independently inoculated into
3-4 germ-free mice by oral gavage. 14 days later mice were orally
infected with 100 spores of C. difficile strain 431. Weight and stool
CFU were monitored for up to 10 days post infection. B) NMDS ordination
of donor stool communities prior to inoculating mice. Each point
represents one donor and are colored by clinical diagnosis. C) NDMS
ordination of the stool communities on day 0. Each symbol represents one
mouse and is colored by donor. Circles represent mice that experienced
mild disease and triangles represent those that suffered severe disease.

\textbf{Figure 2. \emph{C. difficile} infection results in mild or
severe disease.} A) \emph{C. difficile CFU} was enumerated by plating of
mouse stool pellets daily. Each point represents a mouse and the lines
represent the median CFU in each group. Error bars are interquartile
ranges. Red lines and points correspond to mice that succumbed to severe
disease whereas black lines and points correspond to mice that had mild
or no disease. B) Mouse weights were recorded and daily percent weight
loss calculated for each mouse. Data is presented as the median of each
group and interquartile ranges. Mice that succumbed to severe infection
typically lost a significant amount of weight by day 1 or 2 post
infection. Red lines correspond to severely ill mice, black to mice with
mild disease.

\textbf{Figure 3. Random Forest prediction of \emph{C. difficile}
colonization level.} A) OTUs above 1\% relative abundance on day 0 were
used to predict median log\textsubscript{10} CFU of \emph{C. difficile}
after colonization. OTUs were chosen such that they were not predictive
of cage or donor. Each point is a mouse colored by cage. B) Partial
dependency plots of the top six predictive OTUs. Line displats the
partial dependence of log\textsubscript{10} CFU on the relative
abundance of each predictive OUT. Each median log\textsubscript{10} CFU
is plotted against its relative abundance for each predictive OTU.

\textbf{Figure 4. Random Forest prediction of CDI severity.} OTUs above
1\% relative abundance on day 0 were used to predict disease severity.
OTUs were chosen such that they were not predictive of cage or donor.
Predictive classification tested via 10-fold (gray), leave-one-cage-out
(purple dashed) or leave-one-mouse-out (blue dashed) models are
displayed in (A). B) Partial dependency plots of most predictive OTUs.
Line displays the partial dependence of log\textsubscript{10} CFU on OTU
relative abundance. Points are the OTU relative abundance of each mouse
colored by outcome (red, severe, black, mild).

\newpage

\subsection*{References}\label{references}
\addcontentsline{toc}{subsection}{References}

\end{document}
