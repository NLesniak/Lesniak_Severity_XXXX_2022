% Options for packages loaded elsewhere
\PassOptionsToPackage{unicode}{hyperref}
\PassOptionsToPackage{hyphens}{url}
%
\documentclass[
  12pt,
]{article}
\usepackage{lmodern}
\usepackage{amssymb,amsmath}
\usepackage{ifxetex,ifluatex}
\ifnum 0\ifxetex 1\fi\ifluatex 1\fi=0 % if pdftex
  \usepackage[T1]{fontenc}
  \usepackage[utf8]{inputenc}
  \usepackage{textcomp} % provide euro and other symbols
\else % if luatex or xetex
  \usepackage{unicode-math}
  \defaultfontfeatures{Scale=MatchLowercase}
  \defaultfontfeatures[\rmfamily]{Ligatures=TeX,Scale=1}
\fi
% Use upquote if available, for straight quotes in verbatim environments
\IfFileExists{upquote.sty}{\usepackage{upquote}}{}
\IfFileExists{microtype.sty}{% use microtype if available
  \usepackage[]{microtype}
  \UseMicrotypeSet[protrusion]{basicmath} % disable protrusion for tt fonts
}{}
\makeatletter
\@ifundefined{KOMAClassName}{% if non-KOMA class
  \IfFileExists{parskip.sty}{%
    \usepackage{parskip}
  }{% else
    \setlength{\parindent}{0pt}
    \setlength{\parskip}{6pt plus 2pt minus 1pt}}
}{% if KOMA class
  \KOMAoptions{parskip=half}}
\makeatother
\usepackage{xcolor}
\IfFileExists{xurl.sty}{\usepackage{xurl}}{} % add URL line breaks if available
\IfFileExists{bookmark.sty}{\usepackage{bookmark}}{\usepackage{hyperref}}
\hypersetup{
  hidelinks,
  pdfcreator={LaTeX via pandoc}}
\urlstyle{same} % disable monospaced font for URLs
\usepackage[margin=1in]{geometry}
\usepackage{graphicx}
\makeatletter
\def\maxwidth{\ifdim\Gin@nat@width>\linewidth\linewidth\else\Gin@nat@width\fi}
\def\maxheight{\ifdim\Gin@nat@height>\textheight\textheight\else\Gin@nat@height\fi}
\makeatother
% Scale images if necessary, so that they will not overflow the page
% margins by default, and it is still possible to overwrite the defaults
% using explicit options in \includegraphics[width, height, ...]{}
\setkeys{Gin}{width=\maxwidth,height=\maxheight,keepaspectratio}
% Set default figure placement to htbp
\makeatletter
\def\fps@figure{htbp}
\makeatother
\setlength{\emergencystretch}{3em} % prevent overfull lines
\providecommand{\tightlist}{%
  \setlength{\itemsep}{0pt}\setlength{\parskip}{0pt}}
\setcounter{secnumdepth}{-\maxdimen} % remove section numbering
\usepackage{booktabs}
\usepackage{longtable}
\usepackage{array}
\usepackage{multirow}
\usepackage{wrapfig}
\usepackage{float}
\usepackage{colortbl}
\usepackage{pdflscape}
\usepackage{tabu}
\usepackage{threeparttable}
\usepackage{threeparttablex}
\usepackage[normalem]{ulem}
\usepackage{makecell}
\usepackage{caption}
\usepackage{hyperref}
\usepackage{helvet} % Helvetica font
\renewcommand*\familydefault{\sfdefault} % Use the sans serif version of the font
\usepackage[T1]{fontenc}
\usepackage[labelfont=bf]{caption}

\usepackage[none]{hyphenat}

\usepackage{setspace}
\doublespacing
\setlength{\parskip}{1em}

\usepackage{lineno}
\pagenumbering{arabic}
\linenumbers

\usepackage{pdfpages}
\floatplacement{figure}{H} % Keep the figure up top of the page
\usepackage{setspace}
\doublespacing
\usepackage{lineno}
\linenumbers
\newlength{\cslhangindent}
\setlength{\cslhangindent}{1.5em}
\newenvironment{cslreferences}%
  {}%
  {\par}

\author{}
\date{\vspace{-2.5em}}

\begin{document}

\hypertarget{the-gut-bacterial-community-potentiates-clostridioides-difficile-infection-severity.}{%
\section{\texorpdfstring{The gut bacterial community potentiates
\emph{Clostridioides difficile} infection
severity.}{The gut bacterial community potentiates Clostridioides difficile infection severity.}}\label{the-gut-bacterial-community-potentiates-clostridioides-difficile-infection-severity.}}

\vspace{30mm}

\textbf{Running title:} Microbiota potentiates \emph{Clostridioides
difficile} infection severity

\vspace{20mm}

Nicholas A. Lesniak\(^1\), Alyxandria M. Schubert\(^1\), Kaitlyn J.
Flynn\(^1\), Jhansi L. Leslie\(^{1,4}\), Hamide Sinani\(^1\), Ingrid L.
Bergin\(^3\), Vincent B. Young\(^{1,2}\), Patrick D.
Schloss\(^{1,\dagger}\)

\vspace{30mm}

\(\dagger\) To whom correspondence should be addressed:
\href{mailto:pschloss@umich.edu}{\nolinkurl{pschloss@umich.edu}}\\
1. Department of Microbiology and Immunology, University of Michigan,
Ann Arbor, MI\\
2. Division of Infectious Diseases, Department of Internal Medicine,
University of Michigan Medical School, Ann Arbor, MI\\
3. Unit for Laboratory Animal Medicine, University of Michigan, Ann
Arbor, MI\\
4. Current affiliation: Department of Medicine, Division of
International Health and Infectious Diseases, University of Virginia
School of Medicine, Charlottesville, Virginia, USA

\newpage

\hypertarget{abstract}{%
\subsection{Abstract}\label{abstract}}

The severity of \emph{Clostridioides difficile} infections (CDI) has
increased over the last few decades. Patient age, white blood cell
count, creatinine levels as well as \emph{C. difficile} ribotype and
toxin genes have been associated with disease severity. However, it is
unclear whether specific members of the gut microbiota associate with
variation in disease severity. The gut microbiota is known to interact
with \emph{C. difficile} during infection. Perturbations to the gut
microbiota are necessary for \emph{C. difficile} to colonize the gut.
The gut microbiota can inhibit \emph{C. difficile} colonization through
bile acid metabolism, nutrient consumption and bacteriocin production.
Here we sought to demonstrate that members of the gut bacterial
communities can also contribute to disease severity. We derived diverse
gut communities by colonizing germ-free mice with different human fecal
communities. The mice were then infected with a single \emph{C.
difficile} ribotype 027 clinical isolate which resulted in moribundity
and histopathologic differences. The variation in severity was
associated with the human fecal community that the mice received.
Generally, bacterial populations with pathogenic potential, such as
\emph{Enterococcus}, \emph{Helicobacter}, and \emph{Klebsiella}, were
associated with more severe outcomes. Bacterial groups associated with
fiber degradation and bile acid metabolism, such as \emph{Anaerotignum},
\emph{Blautia}, \emph{Lactonifactor}, and \emph{Monoglobus}, were
associated with less severe outcomes. These data indicate that, in
addition to the host and \emph{C. difficile} subtype, populations of gut
bacteria can influence CDI disease severity.

\hypertarget{importance}{%
\subsection{Importance}\label{importance}}

\emph{Clostridioides difficile} colonization can be asymptomatic or
develop into an infection, ranging in severity from mild diarrhea to
toxic megacolon, sepsis, and death. Models that predict severity and
guide treatment decisions are based on clinical factors and \emph{C.
difficile} characteristics. Although the gut microbiome plays a role in
protecting against CDI, its effect on CDI disease severity is unclear
and has not been incorporated into disease severity models. We
demonstrated that variation in the microbiome of mice colonized with
human feces yielded a range of disease outcomes. These results revealed
groups of bacteria associated with both severe and mild \emph{C.
difficile} infection outcomes. Gut bacterial community data from
patients with CDI could improve our ability to identify patients at risk
of developing more severe disease and improve interventions which target
\emph{C. difficile} and the gut bacteria to reduce host damage.

\newpage

\hypertarget{introduction}{%
\subsection{Introduction}\label{introduction}}

\emph{Clostridioides difficile} infections (CDI) have increased in
incidence and severity since \emph{C. difficile} was first identified as
the cause of antibiotic-associated pseudomembranous colitis (1). CDI
disease severity can range from mild diarrhea to toxic megacolon and
death. The Infectious Diseases Society of America (IDSA) and Society for
Healthcare Epidemiology of America (SHEA) guidelines define severe CDI
in terms of a white blood cell count greater than 15,000 cells/mm\(^3\)
and/or a serum creatinine greater than 1.5 mg/dL. Patients who develop
shock or hypotension, ileus, or toxic megacolon are considered to have
fulminant CDI (2). Since these measures are CDI outcomes, they have
limited ability to predict risk of severe CDI when the infection is
first detected. Schemes have been developed to score a patient's risk
for severe CDI outcomes based on clinical factors but have not been
robust for broad application (3). Thus, we have limited ability to
prevent patients from developing severe CDI.

Missing from CDI severity prediction models are the effects of the
indigenous gut bacteria. \emph{C. difficile} interacts with the gut
community in many ways. The indigenous bacteria of a healthy intestinal
community prevent \emph{C. difficile} from infecting the gut (4). A
range of mechanisms can disrupt this inhibition, including antibiotics,
medications, or dietary changes, and lead to increased susceptibility to
CDI (5--7). Once \emph{C. difficile} overcomes the inhibition and
colonizes the intestine, the indigenous bacteria can either promote or
inhibit \emph{C. difficile} through producing molecules or modifying the
environment (8, 9). Bile acids metabolized by the gut bacteria can
inhibit \emph{C. difficile} growth and affect toxin production (4, 10,
11). Bacteria in the gut also can compete more directly with \emph{C.
difficile} through antibiotic production or nutrient consumption
(12--14). While the relationship between the gut bacteria and \emph{C.
difficile} has been established, the effect the gut bacteria can have on
CDI disease severity is unclear.

Recent studies have demonstrated that when mice with diverse microbial
communities were challenged with a high-toxigenic strain resulted in
varied disease severity (15) and when challenged with a low-toxigenic
strain members of the gut microbial community associated with variation
in colonization (16). Here, we sought to further elucidate the
relationship between members of the gut bacterial community and CDI
disease severity when challenged with a high-toxigenic strain, \emph{C.
difficile} ribotype 027 (RT027). We hypothesized that since specific
groups of gut bacteria affect the metabolism of \emph{C. difficile} and
its clearance rate, specific groups of bacteria associate with variation
in CDI disease severity. To test this hypothesis, we colonized germ-free
C57BL/6 mice with human fecal samples to create varied gut communities.
We then challenged the mice with \emph{C. difficile} RT027 and followed
the mice for the development of severe outcomes of moribundity and
histopathologic cecal tissue damage. Since the murine host and \emph{C.
difficile} isolate were the same and only the gut community varied, the
variation in disease severity we observed was attributable to the gut
microbiome.

\hypertarget{results}{%
\subsection{Results}\label{results}}

\textbf{\emph{C. difficile} is able to infect germ-free mice colonized
with human fecal microbial communities without antibiotics.} To produce
gut microbiomes with greater variation than those found in conventional
mouse colonies, we colonized germ-free mice with bacteria from human
feces (17). We inoculated germ-free C57BL/6 mice with homogenized feces
from each of 15 human fecal samples via oral gavage. These human fecal
samples were selected because they represented diverse community
structures based on community clustering (18). After the gut communities
had colonized for two weeks, we confirmed them to be \emph{C. difficile}
negative by culture (19). We then surveyed the bacterial members of the
gut communities by 16S rRNA gene sequencing of murine fecal pellets
(Figure 1A). The bacterial communities from each mouse grouped more
closely to those communities from mice that received the same human
fecal donor community than to the mice who received a different human
fecal donor community (Figure 1B). The communities were primarily
composed of populations of \emph{Clostridia}, \emph{Bacteroidia},
\emph{Erysipelotrichia}, \emph{Bacilli}, and \emph{Gammaproteobacteria}.
However, the gut bacterial communities of each donor group of mice
harbored unique relative abundance distributions of the shared bacterial
classes.

Next, we tested this set of mice with their human-derived gut microbial
communities for susceptibility to \emph{C. difficile} infection. A
typical mouse model of CDI requires pre-treatment of conventional mice
with antibiotics, such as clindamycin, to become susceptible to \emph{C.
difficile} colonization (20, 21). However, we wanted to avoid modifying
the gut communities with an antibiotic to maintain their unique
microbial compositions and ecological relationships. Since some of these
communities came from people at increased risk of CDI, such as recent
hospitalization or antibiotic use (18), we tested whether \emph{C.
difficile} was able to infect these mice without an antibiotic
perturbation. We hypothesized that \emph{C. difficile} would be able to
colonize the mice who received their gut communities from a donor with a
perturbed community. Mice were challenged with 10\(^{3}\) \emph{C.
difficile} RT027 clinical isolate spores. The mice were followed for 10
days post-challenge, and their stool was collected and plated for
\emph{C. difficile} colony forming units (CFU) to determine the extent
of the infection. Surprisingly, communities from all donors were able to
be colonized (Figure 2). Two mice were able to resist \emph{C.
difficile} colonization, both received their community donor N1, which
may be attributed to experimental variation since this group also had
more mice. By colonizing germ-free mice with different human fecal
communities, we were able to generate diverse gut communities in mice,
which were susceptible to \emph{C. difficile} infection without further
modification of the gut community.

\textbf{Infection severity varies by initial community.} After we
challenged the mice with \emph{C. difficile}, we investigated the
outcome from the infection and its relationship to the initial
community. We followed the mice for 10 days post-challenge for
colonization density, toxin production, and mortality. Seven mice, from
Donors N1, N3, N4, and N5, were not colonized at detectable levels on
the day after \emph{C. difficile} challenge but were infected
(\textgreater10\(^{6}\)) by the end of the experiment. All mice that
received their community from Donor M1 through M6 succumbed to the
infection and became moribund within 3 days post-challenge. The
remaining mice, except the uninfected Donor N1 mice, maintained \emph{C.
difficile} infection through the end of the experiment (Figure 2). At 10
days post-challenge, or earlier for the moribund mice, mice were
euthanised and fecal material were assayed for toxin activity and cecal
tissue was collected and scored for histopathologic signs of disease
(Figure 3). Overall, there was greater toxin activity detected in the
stool of the moribund mice (Figure S1). However, when looking at each
group of mice, we observed a range in toxin activity for both the
moribund and non-moribund mice (Figure 3A). Non-moribund mice from
Donors N2 and N5 through N9 had comparable toxin activity as the
moribund mice at 2 days post-challenge. Additionally, not all moribund
mice had toxin activity detected in their stool. Next, we examined the
cecal tissue for histopathologic damage. Moribund mice had high levels
of epithelial damage, tissue edema, and inflammation (Figure S2) similar
to previously reported histopathologic findings for \emph{C. difficile}
RT027 (22). As observed with toxin activity, the moribund mice had
higher histopathologic scores than the non-moribund mice (\emph{P}
\textless{} 0.001). However, unlike the toxin activity, all moribund
mice had consistently high histopathologic summary scores (Figure 3B).
The non-moribund mice, Donor groups N1 through N9, had a range in tissue
damage from none detected to similar levels as the moribund mice, which
grouped by community donor. Together, the toxin activity,
histopathologic score, and moribundity showed variation across the donor
groups but were largely consistent within each donor group.

\textbf{Microbial community members explain variation in CDI severity.}
We next interrogated the bacterial communities at the time of \emph{C.
difficile} challenge (day 0) for their relationship to infection
outcomes using linear discriminant analysis (LDA) effect size (LEfSe)
analysis to identify individual bacterial populations that could explain
the variation in disease severity. We split the mice into groups by
severity level based on moribundity or 10 days post infection (dpi)
histopathologic score for non-moribund. This analysis revealed bacterial
operational taxonomic units (OTUs) that were significantly different at
the time of challenge by the disease severity (Figure 4A). OTUs
associated with \emph{Akkermansia}, \emph{Bacteroides},
\emph{Clostridium sensu stricto}, and \emph{Turicibacter} were detected
at higher relative abundances in the mice that became moribund. OTUs
associated with \emph{Anaerotignum}, \emph{Enterocloster}, and
\emph{Murimonas} were more abundant in the non-moribund mice that would
develop only low intestinal injury. To understand the role of toxin
activity in disease severity, we applied LEfSe to identify the OTUs at
the time of challenge that most likely explain the differences between
communities that had toxin activity detected at anytime point to those
that did not (Figure 4B). An OTU associated with \emph{Bacteroides}, OTU
7, associated with the presence of toxin also associated with
moribundity. Likewise, OTUs associated with \emph{Enterocloster} and
\emph{Murimonas} that were associated with no detected toxin also
exhibited greater relative abundance in communities from non-moribund
mice with a low histopathologic score. Lastly, we tested for
correlations between the endpoint (10 dpi) relative abundances of OTUs
and the histopathologic summary score (Figure 4C). The endpoint relative
abundance of \emph{Bacteroides}, OTU 17, was positively correlated with
histopathologic score, as its day 0 relative abundance did with disease
severity (Figure 4A). A population of \emph{Bacteroides}, OTU 17, was
positively correlated with the histopathologic score and were increased
in the group of mice with detectable toxin. We also tested for
correlations between the endpoint relative abundances of OTUs and toxin
activity but none were significant. This analysis identified bacterial
populations that were associated with the variation in moribundity,
histopathologic score, and toxin.

We next determined whether, collectively, bacterial community membership
and relative abundance could be predictive of the CDI disease outcome.
We trained logistic regression models with bacterial community relative
abundance data from the day of colonization at each taxonomic rank to
predict toxin, moribundity, and histopathologic summary score. For
predicting if detectable toxin would be produced, microbial populations
aggregated by genus rank classification performed similarly as models
using lower taxonomic ranks (mean AUROC = 0.787, Figure S3). \emph{C.
difficile} increased odds of producing detectable toxin when the
community infected had less abundant populations of \emph{Monoglobus},
\emph{Akkermansia}, \emph{Extibacter}, \emph{Intestinimonas} and
\emph{Holdemania} and had more abundant populations of
\emph{Lachnospiraceae} (Figure 5A). Next, we assessed the ability of the
community to predict moribundity. Bacteria grouped by order rank
classification was sufficient to predict which mice would succumb to the
infection before the end of the experiment (mean AUROC = 0.9205, Figure
S3). Many populations contributed to an increase odds of moribundity
(Figure 5B). Populations related to \emph{Bifidobacteriales} and
\emph{Clostridia} decreased the odds of a moribund outcome. Lastly, the
relative abundances of OTUs were able to predict a high or low
histopathologic score 10 dpi (histopathologic scores were dichotomized
as in previous analysis, mean AUROC = 0.99, Figure S3). The model
identified some similar OTUs as the LEfSe analysis, such as
\emph{Murimonas} (OTU 48), \emph{Bacteroides} (OTU 7), and
\emph{Hungatella} (OTU 24). These models have shown that the relative
abundance of bacterial populations and their relationship to each other
could be used to predict the variation in moribundity, histopathologic
score, and detectable toxin of CDI.

\hypertarget{discussion}{%
\subsection{Discussion}\label{discussion}}

Challenging mice colonized with different human fecal communities with
\emph{C. difficile} RT027 demonstrated that variation in members of the
gut microbiome affects \emph{C. difficile} infection disease severity.
Our analysis revealed an association between the relative abundance of
bacterial community members and disease severity. Previous studies
investigating the severity of CDI disease involving the microbiome have
had limited ability to interrogate this relationship between the
microbiome and disease severity. Studies that have used clinical data
have limited ability to control variation in the host, microbiome or
\emph{C. difficile} ribotype (23). Murine experiments typically use a
single mouse colony and different \emph{C. difficile} ribotypes to
create severity differences (24). Recently, our group has begun
uncovering the effect microbiome variation has on \emph{C. difficile}
infection. We showed the variation in the bacterial communities between
mice from different mouse colonies resulted in different clearance rates
of \emph{C. difficile} (16). We also showed varied ability of mice to
spontaneously eliminate \emph{C. difficile} infection when they were
treated with different antibiotics prior to \emph{C. difficile}
challenge (25). Overall, the results presented here have demonstrated
that the gut bacterial community contributed to the severity of \emph{C.
difficile} infection.

\emph{C. difficile} can lead to asymptomatic colonization or infections
with severity ranging from mild diarrhea to death. Physicians use
classification tools to identify patients most at risk of developing a
severe infection using white blood cell counts, serum albumin level, or
serum creatinine level (2, 26, 27). Those levels are driven by the
activities in the intestine (28). Research into the drivers of this
variation have revealed factors that make \emph{C. difficile} more
virulent. Strains are categorized for their virulence by the presence
and production of the toxins TcdA, TcdB, and binary toxin and the
prevalence in outbreaks, such as ribotypes 027 and 078 (20, 29--32).
However, other studies have shown that disease is not necessarily linked
with toxin production (33) or the strain (34). Furthermore, there is
variation in the genome, growth rate, sporulation, germination, and
toxin production in different isolates of a strain (35--38). This
variation may help explain why severe CDI prediction tools often miss
identifying many patients with CDI that will develop severe disease (3,
24, 39, 40). Therefore, it is necessary to gain a full understanding of
all factors contributing to disease variation to improve our ability to
predict severity.

The state of the gut bacterial community determines the ability of
\emph{C. difficile} to colonize and persist in the intestine. \emph{C.
difficile} is unable to colonize an unperturbed healthy murine gut
community and is only able to become established after a perturbation
(21). Once colonized, the different communities lead to different
metabolic responses and dynamics of the \emph{C. difficile} population
(9, 25, 41). Gut bacteria metabolize primary bile acids into secondary
bile acids (4, 42, 43). The concentration of these bile acids affects
germination, growth, toxin production and biofilm formation (10, 11, 44,
45). Members of the bacterial community also affect other metabolites
\emph{C. difficile} utilizes. \emph{Bacteroides thetaiotaomicron}
produce sialidases which release sialic acid from the mucosa for
\emph{C. difficile} to utilize (46, 47). The nutrient environment
affects toxin production (48). Thus, many of the actions of the gut
bacteria modulate \emph{C. difficile} in ways that could affect the
infection and resultant disease.

A myriad of studies have explored the relationship between the
microbiome and CDI disease. Studies examining difference in disease
often use different \emph{C. difficile} strains or ribotypes in mice
with similar microbiota as a proxy for variation in disease, such as
strain 630 for non-severe and RT027 for severe (20, 29, 30, 49). Studies
have also demonstrated variation in infection through tapering
antibiotic dosage (21, 25, 50) or by reducing the amount of \emph{C.
difficile} cells or spores used for the challenge (20, 50). These
studies often either lack variation in the initial microbiome or have
variation in the \emph{C. difficile} infection itself, confounding any
association between variation in severity and the microbiome. Recent
studies have shown variation in the initial microbiome, via different
murine colonies or colonizing germ-free mice with human feces, that were
challenged with \emph{C. difficile} resulted in varied outcomes of the
infection (15, 16, 51).

Our data have demonstrated gut bacterial relative abundances associate
with variation in toxin production, histopathologic scoring of the cecal
tissue and mortality. This analysis revealed populations of
\emph{Akkermansia}, \emph{Anaerotignum}, \emph{Blautia},
\emph{Enterocloster}, \emph{Lactonifactor}, and \emph{Monoglobus} were
more abundant in the microbiome of non-moribund mice which had low
histopathologic scores and no detected toxin. The protective role of
these bacteria are supported by previous studies. \emph{Blautia},
\emph{Lactonifactor}, and \emph{Monoglobus} have been shown to be
involved in dietary fiber fermentation and associated with healthy
communities (52--54). \emph{Anaerotignum}, which produce short chain
fatty acids, has been associated with healthy communities (55, 56).
\emph{Akkermansia} and \emph{Enterocloster} were also identified as more
abundant in mice which had a low histopathologic scores but have
contradictory supporting evidence in the current literature. In our
data, a population of \emph{Akkermansia}, OTU 5, was most abundant in
the non-moribund mice with low histopathologic scores but moribund mice
had increased population of \emph{Akkermansia}, OTU 8. This difference
could indicate either a more protective mucus layer was present
inhibiting colonization (57, 58) or mucus consumption by
\emph{Akkermansia} could have been crossfeeding \emph{C. difficile} or
exposing a niche for \emph{C. difficile} (59--61). Similarly,
\emph{Enterocloster} was more abundant and associated with low
histopathologic scores. It has been associated with healthy populations
and has been used to mono-colonize germ-free mice to reduce the ability
of \emph{C. difficile} to colonize (62, 63). However,
\emph{Enterocloster} has also been involved in infections, such as
bacteremia (64, 65). These data have exemplified populations of bacteria
that have the potential to be either protective or harmful. Thus, the
disease outcome is not likely based on the abundance of individual
populations of bacteria, rather it is the result of the interactions of
the community.

The groups of bacteria that were associated with either a higher
histopathologic score or moribundity are members of the indigenous gut
community that also have been associated with disease, often referred to
as opportunistic pathogens. Some of the populations of
\emph{Bacteroides}, \emph{Enterococcus}, and \emph{Klebsiella} that
associated with worse outcomes, have been shown to have pathogenic
potential, expand after antibiotic use, and are commonly detected in CDI
cases (66--69). In addition to these populations, \emph{Eggerthella},
\emph{Prevotellaceae} and \emph{Helicobacter}, which associated with
worse outcomes, have also been associated with intestinal inflammation
(70--72). Recently, \emph{Helicobacter hepaticus} was shown to be
sufficient to cause susceptibility to CDI in IL-10 deficient C57BL/6
mice (73). In our experiments, when \emph{Helicobacter} was present, the
infection resulted in a high histopathologic score (Figure 4C). While we
did not use IL-10 deficient mice, it is possible the bacterial community
or host response are similarly modified by \emph{Helicobacter}, allowing
\emph{C. difficile} infection and host damage. These bacteria groups
increased in severe outcomes maintained their differences throughout the
length of the experiment (Figure S4). These results agreed Aside from
\emph{Helicobacter}, these groups of bacteria that associated with more
severe outcomes did not have a conserved association between their
relative abundance and the disease severity across all mice.

Since we observed groups of bacteria that were associated with less
severe disease it may be appropriate to apply the damage-response
framework for microbial pathogenesis to CDI (74, 75). This framework
posits that disease is not driven by a single entity, rather it is an
emergent property of the responses of the host immune system, infecting
microbe, \emph{C. difficile}, and the indigenous microbes at the site of
infection. In the first set of experiments, we used the same host
background, C57BL/6 mice, the same infecting microbe, \emph{C.
difficile} RT027 clinical isolate 431, with different gut bacterial
communities. The bacterial groups in those communities were often
present in both moribund and non-moribund and across the range of
histopathologic scores. Thus, it was not merely the presence of the
bacteria but their activity in response to the other microbes and host
which affect the extent of the host damage. Additionally, while each
mouse and \emph{C. difficile} population had the same genetic
background, they too were reacting to the specific microbial community.
Different gut microbial communities can also have different effects on
the host immune responses (76). Disease severity is driven by the
cumulative effect of the host immune response and the activity of
\emph{C. difficile} and the gut bacteria. \emph{C. difficile} drives
host damage through the production of toxin. The gut microbiota can
modulate host damage through the balance of metabolic and competitive
interactions with \emph{C. difficile}, such as bacteriocin production or
mucin degradation, and interactions with the host, such as host mucus
glycosylation or intestinal IL-33 expression (15, 77). For example, low
levels of mucin degradation can provide nutrients to other community
members producing a diverse non-damaging community (78). However, if
mucin degradation becomes too great it reduces the protective function
of the mucin layer and exposes the epithelial cells. This
over-harvesting can contribute to the host damage due to other members
producing toxin. Thus, the resultant intestinal damage is the balance of
all activities in the gut environment. Host damage is the emergent
property of numerous damage-response curves, such as one for host immune
response, one for \emph{C. difficile} activity and another for
microbiome community activity, each of which are a composite curve of
the individual activities from each group, such as antibody production,
neutrophil infiltration, toxin production, sporulation, fiber and mucin
degradation. Therefore, while we have identified populations of
interest, it may be necessary to target multiple types of bacteria to
reduce the community interactions contributing to host damage.

Here we have shown several bacterial groups and their relative
abundances associated with variation in CDI disease severity. Further
understanding how the microbiome affects severity in patients could
reduce the amount of adverse CDI outcomes. When a patient is diagnosed
with CDI, the gut community composition, in addition to the
traditionally obtained clinical information, may improve our severity
prediction and guide prophylactic treatment. Treating the microbiome at
the time of diagnosis, in addition to \emph{C. difficile}, may prevent
the infection from becoming more severe.

\hypertarget{materials-and-methods}{%
\subsection{Materials and Methods}\label{materials-and-methods}}

\textbf{Animal care.} 6- to 13-week old male and female germ-free
C57BL/6 were obtained from a single breeding colony in the University of
Michigan Germ-free Mouse Core. Mice (M1 n=3, M2 n=3, M3 n=3, M4 n=3, M5
n=7, M6 n=3, N1 n=11, N2 n=7, N3 n=3, N4 n=3, N5 n=3, N6 n=3, N7 n=7, N8
n=3, N9 n=2) were housed in cages of 2-4 mice per cage and maintained in
germ-free isolators at the University of Michigan germ-free facility.
All mouse experiments were approved by the University Committee on Use
and Care of Animals at the University of Michigan.

\textbf{\emph{C. difficile} experiments.} Human fecal samples were
obtained as part of Schubert \emph{et al.} and selected based on
community clusters (18) to result in diverse community structures (Table
S1). Feces were homogenized by mixing 200 mg of sample with 5 ml of PBS.
Mice were inoculated with 100 \(\mu\)l of the fecal homogenate via oral
gavage. Two weeks after the fecal community inoculation, mice were
challenged with \emph{C. difficile}. Stool samples from each mouse were
collected one day prior to \emph{C. difficile} and plated for \emph{C.
difficile} enumeration to confirm no \emph{C. difficile} was detected in
stool prior to challenge. \emph{C. difficile} clinical isolate 431 came
from Carlson \emph{et al.} which had previously been isolated and
characterized (35, 36) and has recently been further characterized (37).
Spores concentration were determined both before and after challenge
(79). \(10^{3}\) \emph{C. difficile} spores were given to each mouse via
oral gavage.

\textbf{Sample collection.} Fecal samples were collected on the day of
\emph{C. difficile} challenge and the following 10 days. Each day, a
fecal sample was collected and a portion was weighed for plating
(approximately 30 mg) and the remaining sample was frozen at
-20\(^\circ\)C. Anaerobically, the weighed fecal samples were serially
diluted in PBS, plated on TCCFA plates, and incubated at 37\(^\circ\)C
for 24 hours. The plates were then counted for the number of colony
forming units (CFU) (80).

\textbf{DNA sequencing.} From the frozen fecal samples, total bacterial
DNA was extracted using MOBIO PowerSoil-htp 96-well soil DNA isolation
kit. We amplified the 16S rRNA gene V4 region and sequenced the
resulting amplicons using an Illumina MiSeq as described previously
(81).

\textbf{Sequence curation.} Sequences were processed with
mothur(v.1.44.3) as previously described (81, 82). In short, we used a
3\% dissimilarity cutoff to group sequences into operational taxonomic
units (OTUs). We used a naive Bayesian classifier with the Ribosomal
Database Project training set (version 18) to assign taxonomic
classifications to each OTU (83). We sequenced a mock community of a
known community composition and 16s rRNA gene sequences. We processed
this mock community with our samples to calculate the error rate for our
sequence curation, which was an error rate of 0.19\%.

\textbf{Toxin cytotoxicity assay.} To prepare the sample for the
activity assay, fecal material was diluted 1:10 weight per volume using
sterile PBS and then filter sterilized through a 0.22-\(\mu\)m filter.
Toxin activity was assessed using a Vero cell rounding-based
cytotoxicity assay as described previously (30). The cytotoxicity titer
was determined for each sample as the last dilution, which resulted in
at least 80\% cell rounding. Toxin titers are reported as the log10 of
the reciprocal of the cytotoxicity titer.

\textbf{Histopathology evaluation.} Mouse cecal tissue was placed in
histopathology cassettes and fixed in 10\% formalin, then stored in 70\%
ethanol. McClinchey Histology Labs, Inc.~(Stockbridge, MI) embedded the
samples in paraffin, sectioned, and created the hematoxylin and
eosin-stained slides. The slides were scored using previously described
criteria by a board-certified veterinary pathologist who was blinded to
the experimental groups (30). Slides were scored as 0-4 for parameters
of epithelial damage, tissue edema, and inflammation and a summary score
of 0-12 was generated by summing the three individual parameter scores.
For non-moribund mice, histopathological summary scores used for LEfSe
and logistic regression were split into high and low groups based on
greater or less than the median summary score of 5 because the had a
bimodal distribution (\emph{P} \textless{} 0.05).

\textbf{Statistical analysis and modeling.} To compare community
structures, we calculated Yue and Clayton dissimilarity matrices
(\(\theta\)\textsubscript{YC}) in mothur (84). For this calculation, we
averaged of 1000 sub-samples, or rarified, samples to 2,107 sequence
reads per sample to limit uneven sampling biases. We tested for
differences in individual taxonomic groups that would explain the
outcome differences with LEfSe (85) in mothur (default parameters, LDA
\textgreater{} 4). We tested for differences in temporal trends through
fitting a linear model to each OTU and testing for differences between
histopathological summary scores with LEfSe (85) in mothur (default
parameters, LDA \textgreater{} 3). Remaining statistical analysis and
data visualization was performed in R (v4.0.5) with the tidyverse
package (v1.3.1). We tested for significant differences in
\(\beta\)-diversity (\(\theta\)\textsubscript{YC}), histopathological
scores, and toxin activity using the Wilcoxon rank sum test,
non-unimodality to non-moribund histopathological summary score using
Hartigans' dip test, and toxin detection in mice using the Pearson's
Chi-square test. We used Spearman's correlation to identify which OTUs
that had a correlation between their relative abundance and the
histopathologic summary score. \emph{P} values were then corrected for
multiple comparisons with a Benjamini and Hochberg adjustment for a type
I error rate of 0.05 (86). We built L2 logistic regression models using
the mikropml package (87). Sequence counts were summed by taxonomic
ranks from day 0 samples, normalized by centering to the feature mean
and scaling by the standard deviation, and features positively or
negatively correlated were collapsed into a single feature. For each L2
logistic regression model, we ran 100 random iterations using values of
1e-0, 1e1, 1e2, 2e2, 3e2, 4e2, 5e2, 6e2, 7e2, 8e2, 9e2, 1e3, 1e4 for the
L2 regularization penalty with a split of 80\% of the data for training
and 20\% of the data for testing. Lastly, we did not compare murine
communities to donor community or clinical data because germ-free mice
colonized with non-murine fecal communities have been shown to more
closely resemble the murine communities than the donor species community
(88). Furthermore, it is not our intention to make any inferences
regarding human associated bacteria and their relationship with human
CDI outcome.

\textbf{Code availability.} Scripts necessary to reproduce our analysis
and this paper are available in an online repository
(\url{https://github.com/SchlossLab/Lesniak_Severity_XXXX_2022}).

\textbf{Sequence data accession number.} All 16S rRNA gene sequence data
and associated metadata are available through the Sequence Read Archive
via accession PRJNA787941.

\hypertarget{acknowledgements}{%
\subsection{Acknowledgements}\label{acknowledgements}}

Thank you to Sarah Lucas and Sarah Tomkovich for critical discussion in
the development and execution of this project. We also thank the
University of Michigan Germ-free Mouse Core for assistance with our
germfree mice, funded in part by U2CDK110768. This work was supported by
several grants from the National Institutes for Health R01GM099514,
U19AI090871, U01AI12455, and P30DK034933. Additionally, NAL was
supported by the Molecular Mechanisms of Microbial Pathogenesis training
grant (NIH T32 AI007528). The funding agencies had no role in study
design, data collection and analysis, decision to publish, or
preparation of the manuscript.

Conceptualization: N.A.L., A.M.S., K.J.F., P.D.S.; Data curation:
N.A.L., K.J.F.; Formal analysis: N.A.L., K.J.F., J.L.L., I.L.B.;
Investigation: N.A.L., A.M.S., H.S., I.L.B., V.B.Y., P.D.S.;
Methodology: N.A.L., A.M.S., K.J.F., J.L.L., H.S., I.L.B., V.B.Y.,
P.D.S.; Resources: N.A.L., A.M.S., P.D.S.; Software: NAL; Visualization:
N.A.L., K.J.F., P.D.S.; Writing - original draft: N.A.L.; Writing -
review \& editing: N.A.L., A.M.S., K.J.F., J.L.L., H.S., I.L.B., V.B.Y.,
P.D.S.; Funding acquisition: V.B.Y.; Project administration: P.D.S.;
Supervision: P.D.S. \newpage

\hypertarget{references}{%
\subsection{References}\label{references}}

\hypertarget{refs}{}
\begin{cslreferences}
\leavevmode\hypertarget{ref-Kelly2008}{}%
1. \textbf{Kelly CP}, \textbf{LaMont JT}. 2008. \emph{Clostridium
difficile} --- more difficult than ever. New England Journal of Medicine
\textbf{359}:1932--1940.
doi:\href{https://doi.org/10.1056/nejmra0707500}{10.1056/nejmra0707500}.

\leavevmode\hypertarget{ref-McDonald2018}{}%
2. \textbf{McDonald LC}, \textbf{Gerding DN}, \textbf{Johnson S},
\textbf{Bakken JS}, \textbf{Carroll KC}, \textbf{Coffin SE},
\textbf{Dubberke ER}, \textbf{Garey KW}, \textbf{Gould CV},
\textbf{Kelly C}, \textbf{Loo V}, \textbf{Sammons JS}, \textbf{Sandora
TJ}, \textbf{Wilcox MH}. 2018. Clinical practice guidelines for
\emph{Clostridium difficile} infection in adults and children: 2017
update by the infectious diseases society of america (IDSA) and society
for healthcare epidemiology of america (SHEA). Clinical Infectious
Diseases \textbf{66}:e1--e48.
doi:\href{https://doi.org/10.1093/cid/cix1085}{10.1093/cid/cix1085}.

\leavevmode\hypertarget{ref-Perry2021}{}%
3. \textbf{Perry DA}, \textbf{Shirley D}, \textbf{Micic D},
\textbf{Patel CP}, \textbf{Putler R}, \textbf{Menon A}, \textbf{Young
VB}, \textbf{Rao K}. 2021. External validation and comparison of
\emph{Clostridioides difficile} severity scoring systems. Clinical
Infectious Diseases.
doi:\href{https://doi.org/10.1093/cid/ciab737}{10.1093/cid/ciab737}.

\leavevmode\hypertarget{ref-Buffie2014}{}%
4. \textbf{Buffie CG}, \textbf{Bucci V}, \textbf{Stein RR},
\textbf{McKenney PT}, \textbf{Ling L}, \textbf{Gobourne A}, \textbf{No
D}, \textbf{Liu H}, \textbf{Kinnebrew M}, \textbf{Viale A},
\textbf{Littmann E}, \textbf{Brink MRM van den}, \textbf{Jenq RR},
\textbf{Taur Y}, \textbf{Sander C}, \textbf{Cross JR}, \textbf{Toussaint
NC}, \textbf{Xavier JB}, \textbf{Pamer EG}. 2014. Precision microbiome
reconstitution restores bile acid mediated resistance to
\emph{Clostridium difficile}. Nature \textbf{517}:205--208.
doi:\href{https://doi.org/10.1038/nature13828}{10.1038/nature13828}.

\leavevmode\hypertarget{ref-Britton2014}{}%
5. \textbf{Britton RA}, \textbf{Young VB}. 2014. Role of the intestinal
microbiota in resistance to colonization by \emph{Clostridium
difficile}. Gastroenterology \textbf{146}:1547--1553.
doi:\href{https://doi.org/10.1053/j.gastro.2014.01.059}{10.1053/j.gastro.2014.01.059}.

\leavevmode\hypertarget{ref-Hryckowian2018}{}%
6. \textbf{Hryckowian AJ}, \textbf{Treuren WV}, \textbf{Smits SA},
\textbf{Davis NM}, \textbf{Gardner JO}, \textbf{Bouley DM},
\textbf{Sonnenburg JL}. 2018. Microbiota-accessible carbohydrates
suppress \emph{Clostridium difficile} infection in a murine model.
Nature Microbiology \textbf{3}:662--669.
doi:\href{https://doi.org/10.1038/s41564-018-0150-6}{10.1038/s41564-018-0150-6}.

\leavevmode\hypertarget{ref-VichVila2020}{}%
7. \textbf{Vila AV}, \textbf{Collij V}, \textbf{Sanna S}, \textbf{Sinha
T}, \textbf{Imhann F}, \textbf{Bourgonje AR}, \textbf{Mujagic Z},
\textbf{Jonkers DMAE}, \textbf{Masclee AAM}, \textbf{Fu J},
\textbf{Kurilshikov A}, \textbf{Wijmenga C}, \textbf{Zhernakova A},
\textbf{Weersma RK}. 2020. Impact of commonly used drugs on the
composition and metabolic function of the gut microbiota. Nature
Communications \textbf{11}.
doi:\href{https://doi.org/10.1038/s41467-019-14177-z}{10.1038/s41467-019-14177-z}.

\leavevmode\hypertarget{ref-Abbas2020}{}%
8. \textbf{Abbas A}, \textbf{Zackular JP}. 2020. Microbe-microbe
interactions during \emph{Clostridioides difficile} infection. Current
Opinion in Microbiology \textbf{53}:19--25.
doi:\href{https://doi.org/10.1016/j.mib.2020.01.016}{10.1016/j.mib.2020.01.016}.

\leavevmode\hypertarget{ref-Jenior2017}{}%
9. \textbf{Jenior ML}, \textbf{Leslie JL}, \textbf{Young VB},
\textbf{Schloss PD}. 2017. \emph{Clostridium difficile} colonizes
alternative nutrient niches during infection across distinct murine gut
microbiomes. mSystems \textbf{2}.
doi:\href{https://doi.org/10.1128/msystems.00063-17}{10.1128/msystems.00063-17}.

\leavevmode\hypertarget{ref-Sorg2008}{}%
10. \textbf{Sorg JA}, \textbf{Sonenshein AL}. 2008. Bile salts and
glycine as cogerminants for \emph{Clostridium difficile} spores. Journal
of Bacteriology \textbf{190}:2505--2512.
doi:\href{https://doi.org/10.1128/jb.01765-07}{10.1128/jb.01765-07}.

\leavevmode\hypertarget{ref-Thanissery2017}{}%
11. \textbf{Thanissery R}, \textbf{Winston JA}, \textbf{Theriot CM}.
2017. Inhibition of spore germination, growth, and toxin activity of
clinically relevant \emph{C. difficile} strains by gut microbiota
derived secondary bile acids. Anaerobe \textbf{45}:86--100.
doi:\href{https://doi.org/10.1016/j.anaerobe.2017.03.004}{10.1016/j.anaerobe.2017.03.004}.

\leavevmode\hypertarget{ref-Aguirre2021}{}%
12. \textbf{Aguirre AM}, \textbf{Yalcinkaya N}, \textbf{Wu Q},
\textbf{Swennes A}, \textbf{Tessier ME}, \textbf{Roberts P},
\textbf{Miyajima F}, \textbf{Savidge T}, \textbf{Sorg JA}. 2021. Bile
acid-independent protection against \emph{Clostridioides difficile}
infection. PLOS Pathogens \textbf{17}:e1010015.
doi:\href{https://doi.org/10.1371/journal.ppat.1010015}{10.1371/journal.ppat.1010015}.

\leavevmode\hypertarget{ref-Kang2019}{}%
13. \textbf{Kang JD}, \textbf{Myers CJ}, \textbf{Harris SC},
\textbf{Kakiyama G}, \textbf{Lee I-K}, \textbf{Yun B-S},
\textbf{Matsuzaki K}, \textbf{Furukawa M}, \textbf{Min H-K},
\textbf{Bajaj JS}, \textbf{Zhou H}, \textbf{Hylemon PB}. 2019. Bile acid
7\(\alpha\)-dehydroxylating gut bacteria secrete antibiotics that
inhibit \emph{Clostridium difficile}: Role of secondary bile acids. Cell
Chemical Biology \textbf{26}:27--34.e4.
doi:\href{https://doi.org/10.1016/j.chembiol.2018.10.003}{10.1016/j.chembiol.2018.10.003}.

\leavevmode\hypertarget{ref-Leslie2021}{}%
14. \textbf{Leslie JL}, \textbf{Jenior ML}, \textbf{Vendrov KC},
\textbf{Standke AK}, \textbf{Barron MR}, \textbf{O'Brien TJ},
\textbf{Unverdorben L}, \textbf{Thaprawat P}, \textbf{Bergin IL},
\textbf{Schloss PD}, \textbf{Young VB}. 2021. Protection from lethal
\emph{Clostridioides difficile} infection via intraspecies competition
for cogerminant. mBio \textbf{12}.
doi:\href{https://doi.org/10.1128/mbio.00522-21}{10.1128/mbio.00522-21}.

\leavevmode\hypertarget{ref-NagaoKitamoto2020}{}%
15. \textbf{Nagao-Kitamoto H}, \textbf{Leslie JL}, \textbf{Kitamoto S},
\textbf{Jin C}, \textbf{Thomsson KA}, \textbf{Gillilland MG},
\textbf{Kuffa P}, \textbf{Goto Y}, \textbf{Jenq RR}, \textbf{Ishii C},
\textbf{Hirayama A}, \textbf{Seekatz AM}, \textbf{Martens EC},
\textbf{Eaton KA}, \textbf{Kao JY}, \textbf{Fukuda S}, \textbf{Higgins
PDR}, \textbf{Karlsson NG}, \textbf{Young VB}, \textbf{Kamada N}. 2020.
Interleukin-22-mediated host glycosylation prevents \emph{Clostridioides
difficile} infection by modulating the metabolic activity of the gut
microbiota. Nature Medicine \textbf{26}:608--617.
doi:\href{https://doi.org/10.1038/s41591-020-0764-0}{10.1038/s41591-020-0764-0}.

\leavevmode\hypertarget{ref-Tomkovich2020}{}%
16. \textbf{Tomkovich S}, \textbf{Stough JMA}, \textbf{Bishop L},
\textbf{Schloss PD}. 2020. The initial gut microbiota and response to
antibiotic perturbation influence \emph{Clostridioides difficile}
clearance in mice. mSphere \textbf{5}.
doi:\href{https://doi.org/10.1128/msphere.00869-20}{10.1128/msphere.00869-20}.

\leavevmode\hypertarget{ref-Nagpal2018}{}%
17. \textbf{Nagpal R}, \textbf{Wang S}, \textbf{Woods LCS},
\textbf{Seshie O}, \textbf{Chung ST}, \textbf{Shively CA},
\textbf{Register TC}, \textbf{Craft S}, \textbf{McClain DA},
\textbf{Yadav H}. 2018. Comparative microbiome signatures and
short-chain fatty acids in mouse, rat, non-human primate, and human
feces. Frontiers in Microbiology \textbf{9}.
doi:\href{https://doi.org/10.3389/fmicb.2018.02897}{10.3389/fmicb.2018.02897}.

\leavevmode\hypertarget{ref-Schubert2014}{}%
18. \textbf{Schubert AM}, \textbf{Rogers MAM}, \textbf{Ring C},
\textbf{Mogle J}, \textbf{Petrosino JP}, \textbf{Young VB},
\textbf{Aronoff DM}, \textbf{Schloss PD}. 2014. Microbiome data
distinguish patients with \emph{Clostridium difficile} infection and
non-\emph{C. difficile}-associated diarrhea from healthy controls. mBio
\textbf{5}.
doi:\href{https://doi.org/10.1128/mbio.01021-14}{10.1128/mbio.01021-14}.

\leavevmode\hypertarget{ref-Gillilland2012}{}%
19. \textbf{Gillilland MG}, \textbf{Erb-Downward JR}, \textbf{Bassis
CM}, \textbf{Shen MC}, \textbf{Toews GB}, \textbf{Young VB},
\textbf{Huffnagle GB}. 2012. Ecological succession of bacterial
communities during conventionalization of germ-free mice. Applied and
Environmental Microbiology \textbf{78}:2359--2366.
doi:\href{https://doi.org/10.1128/aem.05239-11}{10.1128/aem.05239-11}.

\leavevmode\hypertarget{ref-Chen2008}{}%
20. \textbf{Chen X}, \textbf{Katchar K}, \textbf{Goldsmith JD},
\textbf{Nanthakumar N}, \textbf{Cheknis A}, \textbf{Gerding DN},
\textbf{Kelly CP}. 2008. A mouse model of \emph{Clostridium
difficile}-associated disease. Gastroenterology \textbf{135}:1984--1992.
doi:\href{https://doi.org/10.1053/j.gastro.2008.09.002}{10.1053/j.gastro.2008.09.002}.

\leavevmode\hypertarget{ref-Schubert2015}{}%
21. \textbf{Schubert AM}, \textbf{Sinani H}, \textbf{Schloss PD}. 2015.
Antibiotic-induced alterations of the murine gut microbiota and
subsequent effects on colonization resistance against \emph{Clostridium
difficile}. mBio \textbf{6}.
doi:\href{https://doi.org/10.1128/mbio.00974-15}{10.1128/mbio.00974-15}.

\leavevmode\hypertarget{ref-Cowardin2016}{}%
22. \textbf{Cowardin CA}, \textbf{Buonomo EL}, \textbf{Saleh MM},
\textbf{Wilson MG}, \textbf{Burgess SL}, \textbf{Kuehne SA},
\textbf{Schwan C}, \textbf{Eichhoff AM}, \textbf{Koch-Nolte F},
\textbf{Lyras D}, \textbf{Aktories K}, \textbf{Minton NP}, \textbf{Petri
WA}. 2016. The binary toxin CDT enhances \emph{Clostridium difficile}
virulence by suppressing protective colonic eosinophilia. Nature
Microbiology \textbf{1}.
doi:\href{https://doi.org/10.1038/nmicrobiol.2016.108}{10.1038/nmicrobiol.2016.108}.

\leavevmode\hypertarget{ref-Seekatz2016}{}%
23. \textbf{Seekatz AM}, \textbf{Rao K}, \textbf{Santhosh K},
\textbf{Young VB}. 2016. Dynamics of the fecal microbiome in patients
with recurrent and nonrecurrent \emph{Clostridium difficile} infection.
Genome Medicine \textbf{8}.
doi:\href{https://doi.org/10.1186/s13073-016-0298-8}{10.1186/s13073-016-0298-8}.

\leavevmode\hypertarget{ref-Dieterle2020}{}%
24. \textbf{Dieterle MG}, \textbf{Putler R}, \textbf{Perry DA},
\textbf{Menon A}, \textbf{Abernathy-Close L}, \textbf{Perlman NS},
\textbf{Penkevich A}, \textbf{Standke A}, \textbf{Keidan M},
\textbf{Vendrov KC}, \textbf{Bergin IL}, \textbf{Young VB}, \textbf{Rao
K}. 2020. Systemic inflammatory mediators are effective biomarkers for
predicting adverse outcomes in \emph{Clostridioides difficile}
infection. mBio \textbf{11}.
doi:\href{https://doi.org/10.1128/mbio.00180-20}{10.1128/mbio.00180-20}.

\leavevmode\hypertarget{ref-Lesniak2021}{}%
25. \textbf{Lesniak NA}, \textbf{Schubert AM}, \textbf{Sinani H},
\textbf{Schloss PD}. 2021. Clearance of \emph{Clostridioides difficile}
colonization is associated with antibiotic-specific bacterial changes.
mSphere \textbf{6}.
doi:\href{https://doi.org/10.1128/msphere.01238-20}{10.1128/msphere.01238-20}.

\leavevmode\hypertarget{ref-Lungulescu2011}{}%
26. \textbf{Lungulescu OA}, \textbf{Cao W}, \textbf{Gatskevich E},
\textbf{Tlhabano L}, \textbf{Stratidis JG}. 2011. CSI: A severity index
for \emph{Clostridium difficile} infection at the time of admission.
Journal of Hospital Infection \textbf{79}:151--154.
doi:\href{https://doi.org/10.1016/j.jhin.2011.04.017}{10.1016/j.jhin.2011.04.017}.

\leavevmode\hypertarget{ref-Zar2007}{}%
27. \textbf{Zar FA}, \textbf{Bakkanagari SR}, \textbf{Moorthi KMLST},
\textbf{Davis MB}. 2007. A comparison of vancomycin and metronidazole
for the treatment of \emph{Clostridium difficile}-associated diarrhea,
stratified by disease severity. Clinical Infectious Diseases
\textbf{45}:302--307.
doi:\href{https://doi.org/10.1086/519265}{10.1086/519265}.

\leavevmode\hypertarget{ref-diMasi2018}{}%
28. \textbf{Masi A di}, \textbf{Leboffe L}, \textbf{Polticelli F},
\textbf{Tonon F}, \textbf{Zennaro C}, \textbf{Caterino M}, \textbf{Stano
P}, \textbf{Fischer S}, \textbf{Hägele M}, \textbf{Müller M},
\textbf{Kleger A}, \textbf{Papatheodorou P}, \textbf{Nocca G},
\textbf{Arcovito A}, \textbf{Gori A}, \textbf{Ruoppolo M}, \textbf{Barth
H}, \textbf{Petrosillo N}, \textbf{Ascenzi P}, \textbf{Bella SD}. 2018.
Human serum albumin is an essential component of the host defense
mechanism against \emph{Clostridium difficile} intoxication. The Journal
of Infectious Diseases \textbf{218}:1424--1435.
doi:\href{https://doi.org/10.1093/infdis/jiy338}{10.1093/infdis/jiy338}.

\leavevmode\hypertarget{ref-AbernathyClose2020}{}%
29. \textbf{Abernathy-Close L}, \textbf{Dieterle MG}, \textbf{Vendrov
KC}, \textbf{Bergin IL}, \textbf{Rao K}, \textbf{Young VB}. 2020. Aging
dampens the intestinal innate immune response during severe
\emph{Clostridioides difficile} infection and is associated with altered
cytokine levels and granulocyte mobilization. Infection and Immunity
\textbf{88}.
doi:\href{https://doi.org/10.1128/iai.00960-19}{10.1128/iai.00960-19}.

\leavevmode\hypertarget{ref-Theriot2011}{}%
30. \textbf{Theriot CM}, \textbf{Koumpouras CC}, \textbf{Carlson PE},
\textbf{Bergin II}, \textbf{Aronoff DM}, \textbf{Young VB}. 2011.
Cefoperazone-treated mice as an experimental platform to assess
differential virulence of \emph{Clostridium difficile} strains. Gut
Microbes \textbf{2}:326--334.
doi:\href{https://doi.org/10.4161/gmic.19142}{10.4161/gmic.19142}.

\leavevmode\hypertarget{ref-Goorhuis2008}{}%
31. \textbf{Goorhuis A}, \textbf{Bakker D}, \textbf{Corver J},
\textbf{Debast SB}, \textbf{Harmanus C}, \textbf{Notermans DW},
\textbf{Bergwerff AA}, \textbf{Dekker FW}, \textbf{Kuijper EJ}. 2008.
Emergence of \emph{Clostridium difficile} infection due to a new
hypervirulent strain, polymerase chain reaction ribotype 078. Clinical
Infectious Diseases \textbf{47}:1162--1170.
doi:\href{https://doi.org/10.1086/592257}{10.1086/592257}.

\leavevmode\hypertarget{ref-OConnor2009}{}%
32. \textbf{O'Connor JR}, \textbf{Johnson S}, \textbf{Gerding DN}. 2009.
\emph{Clostridium difficile} infection caused by the epidemic
BI/NAP1/027 strain. Gastroenterology \textbf{136}:1913--1924.
doi:\href{https://doi.org/10.1053/j.gastro.2009.02.073}{10.1053/j.gastro.2009.02.073}.

\leavevmode\hypertarget{ref-Rao2015}{}%
33. \textbf{Rao K}, \textbf{Micic D}, \textbf{Natarajan M},
\textbf{Winters S}, \textbf{Kiel MJ}, \textbf{Walk ST}, \textbf{Santhosh
K}, \textbf{Mogle JA}, \textbf{Galecki AT}, \textbf{LeBar W},
\textbf{Higgins PDR}, \textbf{Young VB}, \textbf{Aronoff DM}. 2015.
\emph{Clostridium difficile} ribotype 027: Relationship to age,
detectability of toxins A or B in stool with rapid testing, severe
infection, and mortality. Clinical Infectious Diseases
\textbf{61}:233--241.
doi:\href{https://doi.org/10.1093/cid/civ254}{10.1093/cid/civ254}.

\leavevmode\hypertarget{ref-Walk2012}{}%
34. \textbf{Walk ST}, \textbf{Micic D}, \textbf{Jain R}, \textbf{Lo ES},
\textbf{Trivedi I}, \textbf{Liu EW}, \textbf{Almassalha LM},
\textbf{Ewing SA}, \textbf{Ring C}, \textbf{Galecki AT}, \textbf{Rogers
MAM}, \textbf{Washer L}, \textbf{Newton DW}, \textbf{Malani PN},
\textbf{Young VB}, \textbf{Aronoff DM}. 2012. \emph{Clostridium
difficile} ribotype does not predict severe infection. Clinical
Infectious Diseases \textbf{55}:1661--1668.
doi:\href{https://doi.org/10.1093/cid/cis786}{10.1093/cid/cis786}.

\leavevmode\hypertarget{ref-Carlson2013}{}%
35. \textbf{Carlson PE}, \textbf{Walk ST}, \textbf{Bourgis AET},
\textbf{Liu MW}, \textbf{Kopliku F}, \textbf{Lo E}, \textbf{Young VB},
\textbf{Aronoff DM}, \textbf{Hanna PC}. 2013. The relationship between
phenotype, ribotype, and clinical disease in human \emph{Clostridium
difficile} isolates. Anaerobe \textbf{24}:109--116.
doi:\href{https://doi.org/10.1016/j.anaerobe.2013.04.003}{10.1016/j.anaerobe.2013.04.003}.

\leavevmode\hypertarget{ref-Carlson2015}{}%
36. \textbf{Carlson PE}, \textbf{Kaiser AM}, \textbf{McColm SA},
\textbf{Bauer JM}, \textbf{Young VB}, \textbf{Aronoff DM}, \textbf{Hanna
PC}. 2015. Variation in germination of \emph{Clostridium difficile}
clinical isolates correlates to disease severity. Anaerobe
\textbf{33}:64--70.
doi:\href{https://doi.org/10.1016/j.anaerobe.2015.02.003}{10.1016/j.anaerobe.2015.02.003}.

\leavevmode\hypertarget{ref-Saund2021}{}%
37. \textbf{Saund K}, \textbf{Pirani A}, \textbf{Lacy B}, \textbf{Hanna
PC}, \textbf{Snitkin ES}. 2021. Strain variation in \emph{Clostridioides
difficile} toxin activity associated with genomic variation at both
PaLoc and non-PaLoc loci.
doi:\href{https://doi.org/10.1101/2021.12.08.471880}{10.1101/2021.12.08.471880}.

\leavevmode\hypertarget{ref-He2010}{}%
38. \textbf{He M}, \textbf{Sebaihia M}, \textbf{Lawley TD},
\textbf{Stabler RA}, \textbf{Dawson LF}, \textbf{Martin MJ},
\textbf{Holt KE}, \textbf{Seth-Smith HMB}, \textbf{Quail MA},
\textbf{Rance R}, \textbf{Brooks K}, \textbf{Churcher C}, \textbf{Harris
D}, \textbf{Bentley SD}, \textbf{Burrows C}, \textbf{Clark L},
\textbf{Corton C}, \textbf{Murray V}, \textbf{Rose G}, \textbf{Thurston
S}, \textbf{Tonder A van}, \textbf{Walker D}, \textbf{Wren BW},
\textbf{Dougan G}, \textbf{Parkhill J}. 2010. Evolutionary dynamics of
\emph{Clostridium difficile} over short and long time scales.
Proceedings of the National Academy of Sciences \textbf{107}:7527--7532.
doi:\href{https://doi.org/10.1073/pnas.0914322107}{10.1073/pnas.0914322107}.

\leavevmode\hypertarget{ref-Butt2013}{}%
39. \textbf{Butt E}, \textbf{Foster JA}, \textbf{Keedwell E},
\textbf{Bell JE}, \textbf{Titball RW}, \textbf{Bhangu A},
\textbf{Michell SL}, \textbf{Sheridan R}. 2013. Derivation and
validation of a simple, accurate and robust prediction rule for risk of
mortality in patients with \emph{Clostridium difficile} infection. BMC
Infectious Diseases \textbf{13}.
doi:\href{https://doi.org/10.1186/1471-2334-13-316}{10.1186/1471-2334-13-316}.

\leavevmode\hypertarget{ref-vanBeurden2017}{}%
40. \textbf{Beurden YH van}, \textbf{Hensgens MPM}, \textbf{Dekkers OM},
\textbf{Cessie SL}, \textbf{Mulder CJJ}, \textbf{Vandenbroucke-Grauls
CMJE}. 2017. External validation of three prediction tools for patients
at risk of a complicated course of \emph{Clostridium difficile}
infection: Disappointing in an outbreak setting. Infection Control \&
Hospital Epidemiology \textbf{38}:897--905.
doi:\href{https://doi.org/10.1017/ice.2017.89}{10.1017/ice.2017.89}.

\leavevmode\hypertarget{ref-Jenior2018}{}%
41. \textbf{Jenior ML}, \textbf{Leslie JL}, \textbf{Young VB},
\textbf{Schloss PD}. 2018. \emph{Clostridium difficile} alters the
structure and metabolism of distinct cecal microbiomes during initial
infection to promote sustained colonization. mSphere \textbf{3}.
doi:\href{https://doi.org/10.1128/msphere.00261-18}{10.1128/msphere.00261-18}.

\leavevmode\hypertarget{ref-Staley2016}{}%
42. \textbf{Staley C}, \textbf{Weingarden AR}, \textbf{Khoruts A},
\textbf{Sadowsky MJ}. 2016. Interaction of gut microbiota with bile acid
metabolism and its influence on disease states. Applied Microbiology and
Biotechnology \textbf{101}:47--64.
doi:\href{https://doi.org/10.1007/s00253-016-8006-6}{10.1007/s00253-016-8006-6}.

\leavevmode\hypertarget{ref-Long2017}{}%
43. \textbf{Long SL}, \textbf{Gahan CGM}, \textbf{Joyce SA}. 2017.
Interactions between gut bacteria and bile in health and disease.
Molecular Aspects of Medicine \textbf{56}:54--65.
doi:\href{https://doi.org/10.1016/j.mam.2017.06.002}{10.1016/j.mam.2017.06.002}.

\leavevmode\hypertarget{ref-Sorg2010}{}%
44. \textbf{Sorg JA}, \textbf{Sonenshein AL}. 2010. Inhibiting the
initiation of \emph{Clostridium difficile} spore germination using
analogs of chenodeoxycholic acid, a bile acid. Journal of Bacteriology
\textbf{192}:4983--4990.
doi:\href{https://doi.org/10.1128/jb.00610-10}{10.1128/jb.00610-10}.

\leavevmode\hypertarget{ref-Dubois2019}{}%
45. \textbf{Dubois T}, \textbf{Tremblay YDN}, \textbf{Hamiot A},
\textbf{Martin-Verstraete I}, \textbf{Deschamps J}, \textbf{Monot M},
\textbf{Briandet R}, \textbf{Dupuy B}. 2019. A microbiota-generated bile
salt induces biofilm formation in \emph{Clostridium difficile}. npj
Biofilms and Microbiomes \textbf{5}.
doi:\href{https://doi.org/10.1038/s41522-019-0087-4}{10.1038/s41522-019-0087-4}.

\leavevmode\hypertarget{ref-Ng2013}{}%
46. \textbf{Ng KM}, \textbf{Ferreyra JA}, \textbf{Higginbottom SK},
\textbf{Lynch JB}, \textbf{Kashyap PC}, \textbf{Gopinath S},
\textbf{Naidu N}, \textbf{Choudhury B}, \textbf{Weimer BC},
\textbf{Monack DM}, \textbf{Sonnenburg JL}. 2013. Microbiota-liberated
host sugars facilitate post-antibiotic expansion of enteric pathogens.
Nature \textbf{502}:96--99.
doi:\href{https://doi.org/10.1038/nature12503}{10.1038/nature12503}.

\leavevmode\hypertarget{ref-Ferreyra2014}{}%
47. \textbf{Ferreyra JA}, \textbf{Wu KJ}, \textbf{Hryckowian AJ},
\textbf{Bouley DM}, \textbf{Weimer BC}, \textbf{Sonnenburg JL}. 2014.
Gut microbiota-produced succinate promotes \emph{C. difficile} infection
after antibiotic treatment or motility disturbance. Cell Host \& Microbe
\textbf{16}:770--777.
doi:\href{https://doi.org/10.1016/j.chom.2014.11.003}{10.1016/j.chom.2014.11.003}.

\leavevmode\hypertarget{ref-MartinVerstraete2016}{}%
48. \textbf{Martin-Verstraete I}, \textbf{Peltier J}, \textbf{Dupuy B}.
2016. The regulatory networks that control \emph{Clostridium difficile}
toxin synthesis. Toxins \textbf{8}:153.
doi:\href{https://doi.org/10.3390/toxins8050153}{10.3390/toxins8050153}.

\leavevmode\hypertarget{ref-Lawley2012}{}%
49. \textbf{Lawley TD}, \textbf{Clare S}, \textbf{Walker AW},
\textbf{Stares MD}, \textbf{Connor TR}, \textbf{Raisen C},
\textbf{Goulding D}, \textbf{Rad R}, \textbf{Schreiber F},
\textbf{Brandt C}, \textbf{Deakin LJ}, \textbf{Pickard DJ},
\textbf{Duncan SH}, \textbf{Flint HJ}, \textbf{Clark TG},
\textbf{Parkhill J}, \textbf{Dougan G}. 2012. Targeted restoration of
the intestinal microbiota with a simple, defined bacteriotherapy
resolves relapsing \emph{Clostridium difficile} disease in mice. PLoS
Pathogens \textbf{8}:e1002995.
doi:\href{https://doi.org/10.1371/journal.ppat.1002995}{10.1371/journal.ppat.1002995}.

\leavevmode\hypertarget{ref-Reeves2011}{}%
50. \textbf{Reeves AE}, \textbf{Theriot CM}, \textbf{Bergin IL},
\textbf{Huffnagle GB}, \textbf{Schloss PD}, \textbf{Young VB}. 2011. The
interplay between microbiome dynamics and pathogen dynamics in a murine
model of \emph{Clostridium difficile} infection. Gut Microbes
\textbf{2}:145--158.
doi:\href{https://doi.org/10.4161/gmic.2.3.16333}{10.4161/gmic.2.3.16333}.

\leavevmode\hypertarget{ref-Battaglioli2018}{}%
51. \textbf{Battaglioli EJ}, \textbf{Hale VL}, \textbf{Chen J},
\textbf{Jeraldo P}, \textbf{Ruiz-Mojica C}, \textbf{Schmidt BA},
\textbf{Rekdal VM}, \textbf{Till LM}, \textbf{Huq L}, \textbf{Smits SA},
\textbf{Moor WJ}, \textbf{Jones-Hall Y}, \textbf{Smyrk T},
\textbf{Khanna S}, \textbf{Pardi DS}, \textbf{Grover M}, \textbf{Patel
R}, \textbf{Chia N}, \textbf{Nelson H}, \textbf{Sonnenburg JL},
\textbf{Farrugia G}, \textbf{Kashyap PC}. 2018. \emph{Clostridioides
difficile} uses amino acids associated with gut microbial dysbiosis in a
subset of patients with diarrhea. Science Translational Medicine
\textbf{10}.
doi:\href{https://doi.org/10.1126/scitranslmed.aam7019}{10.1126/scitranslmed.aam7019}.

\leavevmode\hypertarget{ref-Liu2021}{}%
52. \textbf{Liu X}, \textbf{Mao B}, \textbf{Gu J}, \textbf{Wu J},
\textbf{Cui S}, \textbf{Wang G}, \textbf{Zhao J}, \textbf{Zhang H},
\textbf{Chen W}. 2021. \emph{Blautia} --- a new functional genus with
potential probiotic properties? Gut Microbes \textbf{13}.
doi:\href{https://doi.org/10.1080/19490976.2021.1875796}{10.1080/19490976.2021.1875796}.

\leavevmode\hypertarget{ref-Mabrok2011}{}%
53. \textbf{Mabrok HB}, \textbf{Klopfleisch R}, \textbf{Ghanem KZ},
\textbf{Clavel T}, \textbf{Blaut M}, \textbf{Loh G}. 2011. Lignan
transformation by gut bacteria lowers tumor burden in a gnotobiotic rat
model of breast cancer. Carcinogenesis \textbf{33}:203--208.
doi:\href{https://doi.org/10.1093/carcin/bgr256}{10.1093/carcin/bgr256}.

\leavevmode\hypertarget{ref-Kim2019}{}%
54. \textbf{Kim CC}, \textbf{Healey GR}, \textbf{Kelly WJ},
\textbf{Patchett ML}, \textbf{Jordens Z}, \textbf{Tannock GW},
\textbf{Sims IM}, \textbf{Bell TJ}, \textbf{Hedderley D},
\textbf{Henrissat B}, \textbf{Rosendale DI}. 2019. Genomic insights from
\emph{Monoglobus pectinilyticus}: A pectin-degrading specialist
bacterium in the human colon. The ISME Journal \textbf{13}:1437--1456.
doi:\href{https://doi.org/10.1038/s41396-019-0363-6}{10.1038/s41396-019-0363-6}.

\leavevmode\hypertarget{ref-Choi2019}{}%
55. \textbf{Choi S-H}, \textbf{Kim J-S}, \textbf{Park J-E}, \textbf{Lee
KC}, \textbf{Eom MK}, \textbf{Oh BS}, \textbf{Yu SY}, \textbf{Kang SW},
\textbf{Han K-I}, \textbf{Suh MK}, \textbf{Lee DH}, \textbf{Yoon H},
\textbf{Kim B-Y}, \textbf{Lee JH}, \textbf{Lee JH}, \textbf{Lee J-S},
\textbf{Park S-H}. 2019. \emph{Anaerotignum faecicola} sp. Nov.,
isolated from human faeces. Journal of Microbiology
\textbf{57}:1073--1078.
doi:\href{https://doi.org/10.1007/s12275-019-9268-3}{10.1007/s12275-019-9268-3}.

\leavevmode\hypertarget{ref-Ueki2017}{}%
56. \textbf{Ueki A}, \textbf{Goto K}, \textbf{Ohtaki Y}, \textbf{Kaku
N}, \textbf{Ueki K}. 2017. Description of \emph{Anaerotignum
aminivorans} gen. Nov., sp. Nov., a strictly anaerobic,
amino-acid-decomposing bacterium isolated from a methanogenic reactor,
and reclassification of \emph{Clostridium propionicum},
\emph{Clostridium neopropionicum} and \emph{Clostridium
lactatifermentans} as species of the genus \emph{anaerotignum}.
International Journal of Systematic and Evolutionary Microbiology
\textbf{67}:4146--4153.
doi:\href{https://doi.org/10.1099/ijsem.0.002268}{10.1099/ijsem.0.002268}.

\leavevmode\hypertarget{ref-Stein2013}{}%
57. \textbf{Stein RR}, \textbf{Bucci V}, \textbf{Toussaint NC},
\textbf{Buffie CG}, \textbf{Rätsch G}, \textbf{Pamer EG}, \textbf{Sander
C}, \textbf{Xavier JB}. 2013. Ecological modeling from time-series
inference: Insight into dynamics and stability of intestinal microbiota.
PLoS Computational Biology \textbf{9}:e1003388.
doi:\href{https://doi.org/10.1371/journal.pcbi.1003388}{10.1371/journal.pcbi.1003388}.

\leavevmode\hypertarget{ref-Nakashima2021}{}%
58. \textbf{Nakashima T}, \textbf{Fujii K}, \textbf{Seki T},
\textbf{Aoyama M}, \textbf{Azuma A}, \textbf{Kawasome H}. 2021. Novel
gut microbiota modulator, which markedly increases \emph{Akkermansia
muciniphila} occupancy, ameliorates experimental colitis in rats.
Digestive Diseases and Sciences.
doi:\href{https://doi.org/10.1007/s10620-021-07131-x}{10.1007/s10620-021-07131-x}.

\leavevmode\hypertarget{ref-Geerlings2018}{}%
59. \textbf{Geerlings S}, \textbf{Kostopoulos I}, \textbf{Vos W de},
\textbf{Belzer C}. 2018. \emph{Akkermansia muciniphila} in the human
gastrointestinal tract: When, where, and how? Microorganisms
\textbf{6}:75.
doi:\href{https://doi.org/10.3390/microorganisms6030075}{10.3390/microorganisms6030075}.

\leavevmode\hypertarget{ref-Deng2018}{}%
60. \textbf{Deng H}, \textbf{Yang S}, \textbf{Zhang Y}, \textbf{Qian K},
\textbf{Zhang Z}, \textbf{Liu Y}, \textbf{Wang Y}, \textbf{Bai Y},
\textbf{Fan H}, \textbf{Zhao X}, \textbf{Zhi F}. 2018. \emph{Bacteroides
fragilis} prevents \emph{Clostridium difficile} infection in a mouse
model by restoring gut barrier and microbiome regulation. Frontiers in
Microbiology \textbf{9}.
doi:\href{https://doi.org/10.3389/fmicb.2018.02976}{10.3389/fmicb.2018.02976}.

\leavevmode\hypertarget{ref-Engevik2020}{}%
61. \textbf{Engevik MA}, \textbf{Engevik AC}, \textbf{Engevik KA},
\textbf{Auchtung JM}, \textbf{Chang-Graham AL}, \textbf{Ruan W},
\textbf{Luna RA}, \textbf{Hyser JM}, \textbf{Spinler JK},
\textbf{Versalovic J}. 2020. Mucin-degrading microbes release
monosaccharides that chemoattract \emph{Clostridioides difficile} and
facilitate colonization of the human intestinal mucus layer. ACS
Infectious Diseases \textbf{7}:1126--1142.
doi:\href{https://doi.org/10.1021/acsinfecdis.0c00634}{10.1021/acsinfecdis.0c00634}.

\leavevmode\hypertarget{ref-Reeves2012}{}%
62. \textbf{Reeves AE}, \textbf{Koenigsknecht MJ}, \textbf{Bergin IL},
\textbf{Young VB}. 2012. Suppression of \emph{Clostridium difficile} in
the gastrointestinal tracts of germfree mice inoculated with a murine
isolate from the family \emph{Lachnospiraceae}. Infection and Immunity
\textbf{80}:3786--3794.
doi:\href{https://doi.org/10.1128/iai.00647-12}{10.1128/iai.00647-12}.

\leavevmode\hypertarget{ref-Ma2021}{}%
63. \textbf{Ma L}, \textbf{Keng J}, \textbf{Cheng M}, \textbf{Pan H},
\textbf{Feng B}, \textbf{Hu Y}, \textbf{Feng T}, \textbf{Yang F}. 2021.
Gut microbiome and serum metabolome alterations associated with isolated
dystonia. mSphere \textbf{6}.
doi:\href{https://doi.org/10.1128/msphere.00283-21}{10.1128/msphere.00283-21}.

\leavevmode\hypertarget{ref-Haas2020}{}%
64. \textbf{Haas KN}, \textbf{Blanchard JL}. 2020. Reclassification of
the \emph{Clostridium clostridioforme} and \emph{Clostridium sphenoides}
clades as \emph{Enterocloster} gen. nov. And \emph{Lacrimispora} gen.
nov., Including reclassification of 15 taxa. International Journal of
Systematic and Evolutionary Microbiology \textbf{70}:23--34.
doi:\href{https://doi.org/10.1099/ijsem.0.003698}{10.1099/ijsem.0.003698}.

\leavevmode\hypertarget{ref-Finegold2005}{}%
65. \textbf{Finegold SM}, \textbf{Song Y}, \textbf{Liu C}, \textbf{Hecht
DW}, \textbf{Summanen P}, \textbf{Könönen E}, \textbf{Allen SD}. 2005.
\emph{Clostridium clostridioforme}: A mixture of three clinically
important species. European Journal of Clinical Microbiology \&
Infectious Diseases \textbf{24}:319--324.
doi:\href{https://doi.org/10.1007/s10096-005-1334-6}{10.1007/s10096-005-1334-6}.

\leavevmode\hypertarget{ref-Tomkovich2021}{}%
66. \textbf{Tomkovich S}, \textbf{Taylor A}, \textbf{King J},
\textbf{Colovas J}, \textbf{Bishop L}, \textbf{McBride K},
\textbf{Royzenblat S}, \textbf{Lesniak NA}, \textbf{Bergin IL},
\textbf{Schloss PD}. 2021. An osmotic laxative renders mice susceptible
to prolonged \emph{Clostridioides difficile} colonization and hinders
clearance. mSphere \textbf{6}.
doi:\href{https://doi.org/10.1128/msphere.00629-21}{10.1128/msphere.00629-21}.

\leavevmode\hypertarget{ref-Keith2020}{}%
67. \textbf{Keith JW}, \textbf{Dong Q}, \textbf{Sorbara MT},
\textbf{Becattini S}, \textbf{Sia JK}, \textbf{Gjonbalaj M},
\textbf{Seok R}, \textbf{Leiner IM}, \textbf{Littmann ER}, \textbf{Pamer
EG}. 2020. Impact of antibiotic-resistant bacteria on immune activation
and \emph{Clostridioides difficile} infection in the mouse intestine.
Infection and Immunity \textbf{88}.
doi:\href{https://doi.org/10.1128/iai.00362-19}{10.1128/iai.00362-19}.

\leavevmode\hypertarget{ref-Zackular2016}{}%
68. \textbf{Zackular JP}, \textbf{Moore JL}, \textbf{Jordan AT},
\textbf{Juttukonda LJ}, \textbf{Noto MJ}, \textbf{Nicholson MR},
\textbf{Crews JD}, \textbf{Semler MW}, \textbf{Zhang Y}, \textbf{Ware
LB}, \textbf{Washington MK}, \textbf{Chazin WJ}, \textbf{Caprioli RM},
\textbf{Skaar EP}. 2016. Dietary zinc alters the microbiota and
decreases resistance to \emph{Clostridium difficile} infection. Nature
Medicine \textbf{22}:1330--1334.
doi:\href{https://doi.org/10.1038/nm.4174}{10.1038/nm.4174}.

\leavevmode\hypertarget{ref-Berkell2021}{}%
69. \textbf{Berkell M}, \textbf{Mysara M}, \textbf{Xavier BB},
\textbf{Werkhoven CH van}, \textbf{Monsieurs P}, \textbf{Lammens C},
\textbf{Ducher A}, \textbf{Vehreschild MJGT}, \textbf{Goossens H},
\textbf{Gunzburg J de}, \textbf{Bonten MJM}, \textbf{Malhotra-Kumar S}.
2021. Microbiota-based markers predictive of development of
\emph{Clostridioides difficile} infection. Nature Communications
\textbf{12}.
doi:\href{https://doi.org/10.1038/s41467-021-22302-0}{10.1038/s41467-021-22302-0}.

\leavevmode\hypertarget{ref-Gardiner2014}{}%
70. \textbf{Gardiner BJ}, \textbf{Tai AY}, \textbf{Kotsanas D},
\textbf{Francis MJ}, \textbf{Roberts SA}, \textbf{Ballard SA},
\textbf{Junckerstorff RK}, \textbf{Korman TM}. 2014. Clinical and
microbiological characteristics of \emph{Eggerthella lenta} bacteremia.
Journal of Clinical Microbiology \textbf{53}:626--635.
doi:\href{https://doi.org/10.1128/jcm.02926-14}{10.1128/jcm.02926-14}.

\leavevmode\hypertarget{ref-Iljazovic2020}{}%
71. \textbf{Iljazovic A}, \textbf{Roy U}, \textbf{Gálvez EJC},
\textbf{Lesker TR}, \textbf{Zhao B}, \textbf{Gronow A}, \textbf{Amend
L}, \textbf{Will SE}, \textbf{Hofmann JD}, \textbf{Pils MC},
\textbf{Schmidt-Hohagen K}, \textbf{Neumann-Schaal M}, \textbf{Strowig
T}. 2020. Perturbation of the gut microbiome by \emph{Prevotella} spp.
enhances host susceptibility to mucosal inflammation. Mucosal Immunology
\textbf{14}:113--124.
doi:\href{https://doi.org/10.1038/s41385-020-0296-4}{10.1038/s41385-020-0296-4}.

\leavevmode\hypertarget{ref-Nagalingam2013}{}%
72. \textbf{Nagalingam NA}, \textbf{Robinson CJ}, \textbf{Bergin IL},
\textbf{Eaton KA}, \textbf{Huffnagle GB}, \textbf{Young VB}. 2013. The
effects of intestinal microbial community structure on disease
manifestation in IL-\(10^{-/-}\) mice infected with \emph{Helicobacter
hepaticus}. Microbiome \textbf{1}.
doi:\href{https://doi.org/10.1186/2049-2618-1-15}{10.1186/2049-2618-1-15}.

\leavevmode\hypertarget{ref-AbernathyClose2021}{}%
73. \textbf{Abernathy-Close L}, \textbf{Barron MR}, \textbf{George JM},
\textbf{Dieterle MG}, \textbf{Vendrov KC}, \textbf{Bergin IL},
\textbf{Young VB}. 2021. Intestinal inflammation and altered gut
microbiota associated with inflammatory bowel disease render mice
susceptible to \emph{Clostridioides difficile} colonization and
infection. mBio.
doi:\href{https://doi.org/10.1128/mbio.02733-20}{10.1128/mbio.02733-20}.

\leavevmode\hypertarget{ref-Pirofski2008}{}%
74. \textbf{Pirofski L-a}, \textbf{Casadevall A}. 2008. The
damage-response framework of microbial pathogenesis and infectious
diseases, pp. 135--146. \emph{In} Advances in experimental medicine and
biology. Springer New York.

\leavevmode\hypertarget{ref-Casadevall2014}{}%
75. \textbf{Casadevall A}, \textbf{Pirofski L-a}. 2014. What is a host?
Incorporating the microbiota into the damage-response framework.
Infection and Immunity \textbf{83}:2--7.
doi:\href{https://doi.org/10.1128/iai.02627-14}{10.1128/iai.02627-14}.

\leavevmode\hypertarget{ref-Lundberg2020}{}%
76. \textbf{Lundberg R}, \textbf{Toft MF}, \textbf{Metzdorff SB},
\textbf{Hansen CHF}, \textbf{Licht TR}, \textbf{Bahl MI}, \textbf{Hansen
AK}. 2020. Human microbiota-transplanted C57BL/6 mice and offspring
display reduced establishment of key bacteria and reduced immune
stimulation compared to mouse microbiota-transplantation. Scientific
Reports \textbf{10}.
doi:\href{https://doi.org/10.1038/s41598-020-64703-z}{10.1038/s41598-020-64703-z}.

\leavevmode\hypertarget{ref-Frisbee2019}{}%
77. \textbf{Frisbee AL}, \textbf{Saleh MM}, \textbf{Young MK},
\textbf{Leslie JL}, \textbf{Simpson ME}, \textbf{Abhyankar MM},
\textbf{Cowardin CA}, \textbf{Ma JZ}, \textbf{Pramoonjago P},
\textbf{Turner SD}, \textbf{Liou AP}, \textbf{Buonomo EL}, \textbf{Petri
WA}. 2019. IL-33 drives group 2 innate lymphoid cell-mediated protection
during \emph{Clostridium difficile} infection. Nature Communications
\textbf{10}.
doi:\href{https://doi.org/10.1038/s41467-019-10733-9}{10.1038/s41467-019-10733-9}.

\leavevmode\hypertarget{ref-Tailford2015}{}%
78. \textbf{Tailford LE}, \textbf{Crost EH}, \textbf{Kavanaugh D},
\textbf{Juge N}. 2015. Mucin glycan foraging in the human gut
microbiome. Frontiers in Genetics \textbf{6}.
doi:\href{https://doi.org/10.3389/fgene.2015.00081}{10.3389/fgene.2015.00081}.

\leavevmode\hypertarget{ref-Sorg2009}{}%
79. \textbf{Sorg JA}, \textbf{Dineen SS}. 2009. Laboratory maintenance
of \emph{Clostridium difficile}. Current Protocols in Microbiology
\textbf{12}.
doi:\href{https://doi.org/10.1002/9780471729259.mc09a01s12}{10.1002/9780471729259.mc09a01s12}.

\leavevmode\hypertarget{ref-Winston2016}{}%
80. \textbf{Winston JA}, \textbf{Thanissery R}, \textbf{Montgomery SA},
\textbf{Theriot CM}. 2016. Cefoperazone-treated mouse model of
clinically-relevant \emph{Clostridium difficile} strain R20291. Journal
of Visualized Experiments.
doi:\href{https://doi.org/10.3791/54850}{10.3791/54850}.

\leavevmode\hypertarget{ref-Kozich2013}{}%
81. \textbf{Kozich JJ}, \textbf{Westcott SL}, \textbf{Baxter NT},
\textbf{Highlander SK}, \textbf{Schloss PD}. 2013. Development of a
dual-index sequencing strategy and curation pipeline for analyzing
amplicon sequence data on the MiSeq illumina sequencing platform.
Applied and Environmental Microbiology \textbf{79}:5112--5120.
doi:\href{https://doi.org/10.1128/aem.01043-13}{10.1128/aem.01043-13}.

\leavevmode\hypertarget{ref-Schloss2009}{}%
82. \textbf{Schloss PD}, \textbf{Westcott SL}, \textbf{Ryabin T},
\textbf{Hall JR}, \textbf{Hartmann M}, \textbf{Hollister EB},
\textbf{Lesniewski RA}, \textbf{Oakley BB}, \textbf{Parks DH},
\textbf{Robinson CJ}, \textbf{Sahl JW}, \textbf{Stres B},
\textbf{Thallinger GG}, \textbf{Horn DJV}, \textbf{Weber CF}. 2009.
Introducing mothur: Open-source, platform-independent,
community-supported software for describing and comparing microbial
communities. Applied and Environmental Microbiology
\textbf{75}:7537--7541.
doi:\href{https://doi.org/10.1128/aem.01541-09}{10.1128/aem.01541-09}.

\leavevmode\hypertarget{ref-Wang2007}{}%
83. \textbf{Wang Q}, \textbf{Garrity GM}, \textbf{Tiedje JM},
\textbf{Cole JR}. 2007. Naïve bayesian classifier for rapid assignment
of rRNA sequences into the new bacterial taxonomy. Applied and
Environmental Microbiology \textbf{73}:5261--5267.
doi:\href{https://doi.org/10.1128/aem.00062-07}{10.1128/aem.00062-07}.

\leavevmode\hypertarget{ref-Yue2005}{}%
84. \textbf{Yue JC}, \textbf{Clayton MK}. 2005. A similarity measure
based on species proportions. Communications in Statistics - Theory and
Methods \textbf{34}:2123--2131.
doi:\href{https://doi.org/10.1080/sta-200066418}{10.1080/sta-200066418}.

\leavevmode\hypertarget{ref-Segata2011}{}%
85. \textbf{Segata N}, \textbf{Izard J}, \textbf{Waldron L},
\textbf{Gevers D}, \textbf{Miropolsky L}, \textbf{Garrett WS},
\textbf{Huttenhower C}. 2011. Metagenomic biomarker discovery and
explanation. Genome Biology \textbf{12}:R60.
doi:\href{https://doi.org/10.1186/gb-2011-12-6-r60}{10.1186/gb-2011-12-6-r60}.

\leavevmode\hypertarget{ref-Benjamini1995}{}%
86. \textbf{Benjamini Y}, \textbf{Hochberg Y}. 1995. Controlling the
false discovery rate: A practical and powerful approach to multiple
testing. Journal of the Royal Statistical Society: Series B
(Methodological) \textbf{57}:289--300.
doi:\href{https://doi.org/10.1111/j.2517-6161.1995.tb02031.x}{10.1111/j.2517-6161.1995.tb02031.x}.

\leavevmode\hypertarget{ref-Topcuoglu2021}{}%
87. \textbf{Topçuoğlu B}, \textbf{Lapp Z}, \textbf{Sovacool K},
\textbf{Snitkin E}, \textbf{Wiens J}, \textbf{Schloss P}. 2021.
Mikropml: User-friendly R package for supervised machine learning
pipelines. Journal of Open Source Software \textbf{6}:3073.
doi:\href{https://doi.org/10.21105/joss.03073}{10.21105/joss.03073}.

\leavevmode\hypertarget{ref-Rawls2006}{}%
88. \textbf{Rawls JF}, \textbf{Mahowald MA}, \textbf{Ley RE},
\textbf{Gordon JI}. 2006. Reciprocal gut microbiota transplants from
zebrafish and mice to germ-free recipients reveal host habitat
selection. Cell \textbf{127}:423--433.
doi:\href{https://doi.org/10.1016/j.cell.2006.08.043}{10.1016/j.cell.2006.08.043}.
\end{cslreferences}

\newpage

\textbf{Figure 1. Human fecal microbial communities established diverse
gut bacterial communities in germ-free mice.} (A) Relative abundances of
the 10 most abundant bacterial classes observed in the feces of
previously germ-free C57Bl/6 mice 14 days post-colonization with human
fecal samples (i.e., day 0 relative to \emph{C. difficile} challenge).
Each column of abundances represents an individual mouse. Mice that
received the same donor feces are grouped together and labeled above
with a letter (N for non-moribund mice and M for moribund mice) and
number (ordered by mean histopathologic score of the donor group). +
indicates the mice which did not have detectable \emph{C. difficile} CFU
(Figure 2). (B) Median (points) and interquartile range (lines) of
\(\beta\)-diversity (\(\theta\)\textsubscript{YC}) between an individual
mouse and either all others which were inoculated with feces from the
same donor or from a different donor. The \(\beta\)-diversity among the
same donor comparison group was significantly less than the
\(\beta\)-diversity of either the different donor group or the donor
community (\emph{P} \textless{} 0.05, calculated by Wilcoxon rank sum
test).

\hfill\break

\textbf{Figure 2. All donor groups resulted in \emph{C. difficile}
infection but with different outcomes.} \emph{C. difficile} CFU per gram
of stool was measured the day after challenge with 10\(^{3}\) \emph{C.
difficile} RT027 clinical isolate 431 spores and at the end of the
experiment, 10 days post-challenge. Each point represents an individual
mouse. Mice are grouped by donor and labeled by the donor letter (N for
non-moribund mice and M for moribund mice) and number (ordered by mean
histopathologic score of the donor group). Points are colored by donor
group. Mice from donor groups N1 through N6 succumbed to the infection
prior to day 10 and were not plated on day 10 post-challenge. LOD =
Limit of detection. -Deceased- indicates mice were deceased at that time
point so no sample was available.

\hfill\break

\textbf{Figure 3. Histopathologic score and toxin activity varied across
donor groups.} (A) Fecal toxin activity was detected in some mice post
\emph{C. difficile} challenge in both moribund and non-moribund mice.
(B) Cecum scored for histopathologic damage from mice at the end of the
experiment. Samples were collected for histopathologic scoring on day 10
post-challenge for non-moribund mice or the day the mouse succumbed to
the infection for the moribund group (day 2 or 3 post-challenge). Each
point represents an individual mouse. Mice are grouped by donor and
labeled by the donor letter (N for non-moribund mice and M for moribund
mice) and number (ordered by mean histopathologic score of the donor
group). Points are colored by donor group. Mice in group N1 that have a
summary score of 0 are the mice which did not have detectable \emph{C.
difficile} CFU (Figure 2). Missing points are from mice that had
insufficient fecal sample collected for assaying toxin or cecum for
histopathologic scoring. * indicates significant difference between
non-moribund and moribund groups of mice by Wilcoxon test (\emph{P}
\textless{} 0.002). LOD = Limit of detection. -Deceased- indicates mice
were deceased at that time point so no sample was available.

\hfill\break

\textbf{Figure 4. Individual fecal bacterial community members of the
murine gut associated with \emph{C .difficile} infection outcomes.} (A
and B) Relative abundance of OTUs at the time of \emph{C. difficile}
challenge (Day 0) that varied significantly by the moribundity and
histopathologic summary score or detected toxin by LEfSe analysis.
Median (points) and interquartile range (lines) are plotted. (A) Day 0
relative abundances were compared across infection outcome of moribund
(colored black) or non-moribund with either a high histopathologic score
(score greater than the median score of 5, colored green) or a low
histopathologic summary score (score less than the median score of 5,
colored light green). (B) Day 0 relative abundances were compared
between mice which toxin activity was detected (Toxin +, colored dark
purple) and which no toxin activity was detected (Toxin -, colored light
purple). (C) Day 10 bacterial OTU relative abundances correlated with
histopathologic summary score. Each individual mouse is plotted and
colored according to their categorization in panel A. Points at the
median score of 5 (gray points) were not included in panel A. Spearman's
correlations were statistically significant after Benjamini-Hochberg
correction for multiple comparisons. All bacterial groups are ordered by
the LDA score. * indicates that the bacterial group was unclassified at
lower taxonomic classification ranks.

\hfill\break

\textbf{Figure 5. Fecal bacterial community members of the murine gut at
the time of \emph{C. difficile} infection predicted outcomes of the
infection.} On the day of infection (Day 0), bacterial community members
grouped by different classification rank were modeled with logistic
regression to predict the infection outcome. The models used the highest
taxonomic classification rank without a decrease in performance. Models
used all community members but plotted are those members with a mean
odds ratio not equal to 1. Median (solid points) and interquartile range
(lines) of the group relative abundance are plotted. Bacterial groups
are ordered by their odds ratio. * indicates that the bacterial group
was unclassified at lower taxonomic classification ranks. (A) Bacterial
members grouped by genus predicted which mice would have toxin activity
detected at any point throughout the infection (Toxin +, dark purple).
(B) Bacterial members grouped by order predicted which mice would become
moribund (dark blue). (C) Bacterial members grouped by OTU predicted if
the mice would have a high (score greater than the median score of 5,
colored dark green) or low (score less than the median score of 5,
colored light green) histopathologic summary score.

\hfill\break

\textbf{Figure S1. Toxin detect in mice based on outcome of the
infection.} Comparison of the distribution of number of either
non-moribund or moribund mice which toxin was detected in the first
three days post infection. Bars are colored by whether toxin was
detected in stool from the mouse (dark purple) or not (light purple).
Moribund mice had significantly more mice with toxin detected (\emph{P}
\textless{} 0.008) by Pearson's Chi-square test.

\hfill\break

\textbf{Figure S2. Histopathologic score of tissue damage at the
endpoint of the infection.} Tissue collected at the endpoint, either day
10 post-challenge (Non-moribund) or day mice succumbed to infection
(Moribund), were scored from histopathologic damage. Each point
represents an individual mouse. Mice (points) are grouped and colored by
their human fecal community donor. Missing points are from mice that had
insufficient sample for histopathologic scoring. * indicates significant
difference between non-moribund and moribund groups of mice by Wilcoxon
test (\emph{P} \textless{} 0.002).

\hfill\break

\textbf{Figure S3. Logistic regression models predicted outcomes of the
\emph{C. difficile} challenge.} (A-C) Taxonomic classification rank
model performance. Relative abundance at the time of \emph{C. difficile}
challenge (Day 0) of the bacterial community members grouped by
different classification rank were modeled with random forest to predict
the infection outcome. The models used the highest taxonomic
classification rank performed as well as the lower ranks. Black
rectangle highlights classification rank used to model each outcome. For
all plots, median (large solid points), interquartile range (lines), and
individual models (small transparent points) are plotted. (A) Toxin
production modeled which mice would have toxin detected during the
experiment. (B) Moribundity modeled which mice would succumb to the
infection prior to day 10 post-challenge. (C) Histopathologic score
modeled which mice would have a high (score greater than the median
score of 5) or low (score less than the median score of 5)
histopathologic summary score.

\hfill\break

\textbf{Figure S4. Temporal dynamics of OTUs that differed between
histopathologic summary score.} Relative abundance of OTUs on each day
relative to the time of \emph{C. difficile} challenge (Day 0) that have
a significantly different temporal trend by the histopathologic summary
score by LEfSe analysis. Median (points) and interquartile range (lines)
are plotted. Points and lines are colored by infection outcome of
moribund (colored black) or non-moribund with either a high
histopathologic score (score greater than the median score of 5, colored
green) or a low histopathologic summary score (score less than the
median score of 5, colored light green).

\end{document}
