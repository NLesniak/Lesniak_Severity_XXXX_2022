\documentclass[11pt,]{article}
\usepackage{lmodern}
\usepackage{amssymb,amsmath}
\usepackage{ifxetex,ifluatex}
\usepackage{fixltx2e} % provides \textsubscript
\ifnum 0\ifxetex 1\fi\ifluatex 1\fi=0 % if pdftex
  \usepackage[T1]{fontenc}
  \usepackage[utf8]{inputenc}
\else % if luatex or xelatex
  \ifxetex
    \usepackage{mathspec}
    \usepackage{xltxtra,xunicode}
  \else
    \usepackage{fontspec}
  \fi
  \defaultfontfeatures{Mapping=tex-text,Scale=MatchLowercase}
  \newcommand{\euro}{€}
\fi
% use upquote if available, for straight quotes in verbatim environments
\IfFileExists{upquote.sty}{\usepackage{upquote}}{}
% use microtype if available
\IfFileExists{microtype.sty}{%
\usepackage{microtype}
\UseMicrotypeSet[protrusion]{basicmath} % disable protrusion for tt fonts
}{}
\usepackage[margin=1.0in]{geometry}
\ifxetex
  \usepackage[setpagesize=false, % page size defined by xetex
              unicode=false, % unicode breaks when used with xetex
              xetex]{hyperref}
\else
  \usepackage[unicode=true]{hyperref}
\fi
\hypersetup{breaklinks=true,
            bookmarks=true,
            pdfauthor={},
            pdftitle={Microbiota predict Clostridium difficile severity in germ-free mice colonized with human feces},
            colorlinks=true,
            citecolor=blue,
            urlcolor=blue,
            linkcolor=magenta,
            pdfborder={0 0 0}}
\urlstyle{same}  % don't use monospace font for urls
\setlength{\parindent}{0pt}
\setlength{\parskip}{6pt plus 2pt minus 1pt}
\setlength{\emergencystretch}{3em}  % prevent overfull lines
\setcounter{secnumdepth}{0}

%%% Use protect on footnotes to avoid problems with footnotes in titles
\let\rmarkdownfootnote\footnote%
\def\footnote{\protect\rmarkdownfootnote}

%%% Change title format to be more compact
\usepackage{titling}

% Create subtitle command for use in maketitle
\newcommand{\subtitle}[1]{
  \posttitle{
    \begin{center}\large#1\end{center}
    }
}

\setlength{\droptitle}{-2em}
  \title{\textbf{Microbiota predict \emph{Clostridium difficile} severity in
germ-free mice colonized with human feces}}
  \pretitle{\vspace{\droptitle}\centering\huge}
  \posttitle{\par}
  \author{}
  \preauthor{}\postauthor{}
  \date{}
  \predate{}\postdate{}

\usepackage{helvet} % Helvetica font
\renewcommand*\familydefault{\sfdefault} % Use the sans serif version of the font
\usepackage[T1]{fontenc}

\usepackage[none]{hyphenat}

\usepackage{setspace}
\doublespacing
\setlength{\parskip}{1em}

\usepackage{lineno}

\usepackage{pdfpages}


\begin{document}

\maketitle


\vspace{35mm}

Running title: Microbiota predict \emph{C. difficile} severity in
humanized mice

\vspace{35mm}

Kaitlin J. Flynn\textsuperscript{1+}, Nicholas
Lesniak\textsuperscript{1+}, Alyxandria M. Schubert\textsuperscript{2},
Hamide Sinani\textsuperscript{?}, and Patrick D.
Schloss\textsuperscript{1\(\dagger\)}

\vspace{40mm}

\(\dagger\) Corresponding author:
\href{mailto:pschloss@umich.edu}{\nolinkurl{pschloss@umich.edu}}

+ These authors contributed equally to this work

1. Department of Microbiology and Immunology, University of Michigan,
Ann Arbor, Michigan 48109

2. Food and Drug Administration?

3. Department for Hamide?

\newpage
\linenumbers

\subsubsection{Abstract}\label{abstract}

\emph{Clostridium difficile} causes diarrheal disease when it
successfully colonizes a dysbiotic gut microbial community. Current
mouse models to study \emph{C. difficile} infection (CDI) rely on
pre-treatment with antibiotics to disrupt the mouse microbiome prior to
inoculation. This model does not allow for analysis of human-associated
microbial community members that modulate \emph{C. difficile}
colonization and expansion. To study human-associated microbes in the
context of CDI, we inoculated germ-free C57BL/6 mice with one of 16
human fecal samples from diarrheal or healthy donors and challenged with
C. difficile 14 days later. Five unique donor-mice combinations resulted
in severe CDI while the remaining 11 only experienced mild disease. Both
healthy and diarrheal donors were susceptible to colonization and severe
symptoms of CDI. To determine if specific microbes were associated with
disease severity outcomes, we built a classification Random Forest
machine learning model based on relative abundance data of the
communities prior to infection. The model identified a number of
bacterial populations associated with the development of severe CDI,
including \emph{Bacilliales, Ruminococcaceae, Ruminococcus,
Staphylococcus, Streptococcus and Bacteriodetes}. Additionally, a
regression model accurately predicted colonization levels of \emph{C.
difficile} at one to ten days post-infection. This model explained 99\%
of the variance in the number of CFU isolated from mouse stool. Members
of \emph{Lachnospiraceae, Parabacteroides, Bacteroidales, Bacteroidetes,
Porphyromonadaceae} and unclassified \emph{Bacteria} families were
predictive of future \emph{C. difficile} colonization levels. Finally,
challenging these mice with different strains of \emph{C. difficile}
revealed that susceptible human-associated microbial communities were
prone to severe disease independent of strain type. Taken together these
results suggest that human-associated microbial communities can be
recapitulated in germ-free mice and used to characterize dynamics of
CDI. Because both healthy and diarrheal patients were susceptible to
severe CDI, machine-learning models are useful to identify bacterial
populations that allow colonization and contribute to the development of
\emph{C. difficile} associated disease in humans.

\newpage

\subsubsection{Introduction}\label{introduction}

\emph{Clostridium difficile} is an opportunistic pathogen of the human
lower gastrointestinal tract. \emph{C. difficile} forms spores that can
persist on abiotic surfaces and are not readily killed by ethanol-based
hand-sanitizers, putting hospital patients particularly at risk. Indeed,
\textasciitilde{}12\% of hospital acquired infections in the United
States are due to \emph{C. difficile} and result in up to 15,000 deaths
annually (2). Disruption of the native microbial community is the most
common risk factor for development of \emph{C. difficile} infection
(CDI) (1). Antibiotic use and inflammatory bowel diseases are associated
with loss of colonization resistance to \emph{C. difficile} through the
loss of potentially protective bacterial families such as
\emph{Barnesiella} and \emph{Lachnospiraceae} in both mouse models and
human association experiments (cite alyx 2014 and vincent 2013). The
composition of the community is clearly important for the acquisition,
resistance and treatment of \emph{C. difficile} infection, as giving
patients a healthy fecal microbiome transplant is the most effective
treatment for this disease (cite Anna 2014). The precise mechanisms of
colonization resistance and restoration of a healthy community are yet
to be discovered.

Murine models to study CDI typically rely on treating
conventionally-raised mice with antibiotics either in drinking water or
by injection to induce susceptibility (3, 4). This model provides a
convenient way to study \emph{C. difficile} pathogenesis and virulence
factors. Numerous microbiome studies have been performed using this
model to determine the antibiotic classes (5), starting microbial
community (6) and metabolites (7) that impact development and severity
of \emph{C. difficile} infection. While informative, these studies are
somewhat removed from human disease because they only examine
mouse-associated microbial communities.

Gnotobiotic or germ-free mouse models have been used for a range of
studies of CDI, including assessment of species-specific interactions
between \emph{C. difficile} and competing microbial community members
(8), analysis of nutrient restriction (9), \emph{in vivo}
transcriptomics of \emph{C. difficile} and examination of host immune
response to CDI (10). Further, CDI therapeutics such as antibiotics and
fecal microbiota transplants have been tested extensively in a
gnotobiotic-piglet or piglet-to-gnotobiotic-mouse model of disease (11),
(12). Pigs have a longer digestive tract with components more similar to
humans than mice and are typically infected by strains typical in human
infection (13). However, the murine and porcine microbiomes typically do
not resemble those of the human gut.

The power of the gnotobiotic models to study CDI has been further
realized by first inoculating germ-free mice and piglets with human
stool microbes. In one study, germ-free piglets were acutely colonized
with human feces for one week and then treated with tigecycline. After
challenge with C. difficile none of the antibiotic-treated piglets
succumbed to infection, while some of the untreated human-colonized pigs
did (11). In another study, germ-free mice colonized with human feces
were bred over several generations to create a cohort of mice with
identical human-derived microbiomes (14). These mice were subsequently
treated with a five-antibiotic cocktail to induce dysbiosis and then
were successfully colonized by \emph{C. difficile} (14). While
informative, these studies were limited in their use of only one human
donor as input inoculum. In order to best understand the impact of
\emph{C. difficile} pathogenesis on human disease, we must have a
laboratory model that allows for study of a variety of human-derived
microbiomes.

Here we designed a laboratory model that allows for studying and
modeling microbial coumminity interactions with \emph{C. difficile}
infection in mice with human derived microbiomes. To test the impact of
individual human microbiomes on CDI, we colonized germ-free mice with 16
different human stool donors. We then characterized human-associated
microbiome response to C. difficile challenge. Additionally, the use of
machine-learning models allowed us to build a predictive model that
classified ``at-risk'' microbiomes prior to infection with C. difficile.
These findings show that human-associated microbiomes can be at risk for
CDI even in the absence of antibiotics and that study of mice colonized
with human feces provides a range of clinical outcomes.

\textbf{also want something in the intro about severity maybe, and
asymptomatic colonization. or in discussion?}

\subsubsection{Results}\label{results}

\textbf{Germ-free mice inoculated with human feces as model for C.
difficile infection.} To generate mice with human-derived microbiomes,
we inoculated one cage of gnotobiotic C57/BL6 mice with one of 16
different human fecal donors. Five donors were patients that had
diarrhea that was not attributable to \emph{C. difficile} infection
while 11 donors were healthy at time of donation. Stool from a patient
that was colonized with virulent \emph{C. difficile} was used as a
positive control. After inoculation with human stool, mice were allowed
to equilibrate for 14 days. Prior to infection, stool samples were taken
from each mouse to establish baseline. Then, the \emph{C. difficile}
strain isolated from the positive control patient's sample (strain 430)
was used to infect each mouse with 100 spores. Mice were monitored for
weight loss and clinical signs of disease. Fecal samples were taken to
enumerate \emph{C. difficile} CFU and for microbiome analysis every day
for up to 10 days post-infection (Fig 1A). To ensure that the donors we
selected represented a diverse array of human microbiomes, we sequenced
the 16S rRNA genes from donor fecal inocula. Ordination of the distances
between donor communities showed that the donors each had distinctly
different communities, independent of whether the sample came from a
sick or healthy person (Fig 1B). Likewise, the starting microbial
communities of the mice on day 0 were characterizing by sequencing of
fecal pellets DNA prior to infection. Ordination of all of the mouse
communities on day 0 shows that mice were similar to each other within
each cage and donor, but distinct from other donors (Fig 1C). This
result confirmed that human-associated microbes were able to colonize
gnotobiotic mice and provide distinct initial communities to test
\emph{C. difficile} dynamics.

\textbf{need to put in a line to reference supplemental table with RA in
it}

\textbf{C. difficile infection in mice with human-derived microbiota
cause a range of outcomes.} \emph{C. difficile} colonization was
monitored by daily plating of stool pellets for \emph{C. difficile} CFU.
Nearly all of the mice were colonized to 10\textsuperscript{5} --
10\textsuperscript{7} CFU by one day post-infection and remained
colonized at that level until the end of the experiment (Fig 2A). As one
indicator of disease, mouse weights were taken each day post-infection
and weight-loss was monitored alongside clinical signs of disease. When
mice were judged to be too ill to continue they were humanely
euthanized. Overall, disease phenotypes fell into two classes. Mice that
became severely ill and lost 20\% or more of their starting body weight
within one to two days post-infection were classified as ``severe''
whereas mice that were colonized with \emph{C. difficile} but did not
show signs of disease or severe weight loss were considered to have
``mild'' disease (Fig 2A, 2B). Interestingly, \emph{C. difficile} was
able to cause severe disease in both mice that had been colonized with
healthy stool and those colonized with diarrheal stool, suggesting
susceptibility to CDI is dependent on the composition of the starting
microbiome and not associated with donor clinical metadata.

\textbf{Microbes present in the gut prior to infection are predictive of
\emph{C. difficile} colonization levels} Previous work in our group has
demonstrated that the microbes present in the mouse microbiome prior to
\emph{C. difficile} challenge can predict future colonization levels
(cite Alyx). To determine if the human-associated microbiome in the
mouse could similarly predict \emph{C. difficile} colonization we
employed a Random Forest machine learning algorithm similar to the one
used in previous studies. Relative abundance of OTUs with greater than
1\% abundance at day 0 were used as input to the model. We found that
the model explained X\% of the variance in colonization levels of
\emph{C. difficile} (Fig 3A). By refining the model to include only the
top X predictive OTUs, the model explained X\% of the variance. Partial
dependency plots of the top 6 OTUs predictive of \emph{C. difficile}
colonization reveal that colonization is positively associated with
\emph{Bacteriodetes}, \emph{Parabacteriodes} and
\emph{Porphyromonadaceae} (really?) (Fig 3B). Colonization was
negatively associated with the presence of \emph{Lachnospiraceae} (Fig
3B) though none of the human-associated mouse communities were
completely resistant to colonization by \emph{C. difficile} (Fig 2A,
3B). These results confirm that as in mouse microbiome studies, the
human-associated gut community prior to challenge is predictive of
\emph{C. difficile} colonization levels.

\textbf{The microbiome predicts \emph{C. difficile} infection severity.}
As \emph{C. difficile} colonization levels can be predicted by the
microbiome composition on day 0, we postulated that severity of disease
could also be predicted from day 0. Using weight loss as a proxy for
severity, we built a Random Forest classification model based on
relative abundance to determine which community members were associated
with either mild or severe disease. We optimized the model using the
AUC-RF algorithm to generate a Reciever Operating Curve (ROC) with
maximal Area Under the Curve (AUC) (cite Calle 2011), as used previously
in our group to classify colon cancer outcomes (cite Niel 2015 gen med).
The optimized model used X number of OTUs and identified several that
predicted increased weightloss and subsequent disease severity (Fig 4A,
4B). The populations that were most commonly found in mice that
succumbed to \emph{C. difficile} infection were members of the
\emph{Clostridiales}, \emph{Bacilliales}, \emph{Streptococcus}
(Lactobacilles?) and \emph{Bacteriodes}. (need more OTUs here, or number
that were used for model and then group by rank). The AUC was calculated
using the Out-Of-Bag (OOB) error for each sample (Fig 4A).
Cross-validation using 10-fold, leave-one-mouse-out and
leave-one-cage-out resulted in AUCs not significantly different from the
OOB (Fig 4A). This model reveals that the levels of X OTUs on day 0 can
predict disease outcome in mice with human-associated microbiomes.
(probably also going to want a supplemental table for predictive OTUs
for this and Fig 3)

\textbf{Susceptibility to infection is community-dependent.} \emph{C.
difficile} strains are known to vary in disease phenotypes based both on
pathogenicity factors of the strain and host microbiome composition
(cite someone). To test the susceptibility of human-derived microbiomes
to \emph{C. difficile} infection, two donor-mouse microbiomes that were
previously shown to be resistant to \emph{C. difficile} strain 431
induced weight loss (Fig 2B) (DA00396, DA00430) and one that was
suscpetible to severe disease (DA00578) were independently challenged
with one of three unique \emph{C. difficile} strains (table or reference
for toxin prod, etc). As done previously, \emph{C. difficile} stool CFU
was enumerated daily and mouse weight-loss tracked. Infection of mice
with resistant communities with virulent strain 458 or the weaker strain
299 resulted in colonization to levels near that of 431 (Fig 5A) and
these mice did not experience severe weight loss or show clinical signs
of disease (Fig 5B). In contrast, infection of mice harboring
microbiomes susceptible to severe disease (DA00578, Fig 1B) with strain
395 resulted in colonization and severe weight loss by two days
post-infection (Fig 5A, 5B). Strain 395 is similarly virulent as strain
431, and shows a similar infection phenotype in these mice. These
preliminary results suggest that human-associated microbiomes may be
more predictive of disease severity than the \emph{C. difficile} strain.

\subsubsection{Discussion}\label{discussion}

Study of human-associated microbiomes and their interactions with
\emph{C. difficile} has been limited by the lack of an appropriate model
and available samples. Here we colonized 16 groups of mice with human
donor stool from patients with different clinical profiles or healthy
volunteers. This created a diverse set of human-associated microbiomes
in the mice that allowed for testing \emph{C. difficile} infection
dynamics and community interactions. Our results showed that both
healthy and diarrheal-associated human communities were susceptible to
severe disease. Additionally, our results confirmed our and others
previous studies that proved that resistance to \emph{C. difficile}
disease is dependent on the entire microbial community and not just one
or two species (cite alyx). Further, we were able to learn
classification models that identified specific OTUs within susceptible
communities that predispose mice to infection. By subsequently infecting
susceptible or resistant communities with varying strains of \emph{C.
difficile} we confirmed that the initial state of the human-associated
microbiome is more predictive of disease severity than the strain used.

Here we were able to colonize germ-free mice with a variety of human
stool communities from diarrheal or healthy people. After allowing these
communities to colonize the mice for 2 weeks, they were challenged with
\emph{C. difficile}. It is important to note that the mouse gut
environment is different from that of the human, and as such exerts a
selective pressure on the human stool microbial community that
stabilizes over the 2 weeks. Thus these mice cannot be considered
``humanized'' but rather containing human-associated microbes.
Interestingly, clinical status was not associated with proclivity to
severe \emph{C. difficile} infection as some healthy donor mice were
susceptible. A surprising finding was that none of the human-associated
mouse communities were completely resistant to colonization by \emph{C.
difficile}. This is in contrast to numerous reports of mouse microbiome
studies (cite) and the knowledge that in general healthy humans without
gut perturbations are resistant to \emph{C. difficile} infection (cite).
On the other hand, it is known that humans, especially health care
employees, can be asymptomatic carriers of \emph{C. difficile}
(Furyura-Kanamari). This may explain the fact that the majority of our
groups of mice were colonized by \emph{C. difficile} but did not display
outward signs of illness or infection-associated weight loss (Fig 2).

• Discuss prediction methods and outcomes - Nick - we have bugs that
predict severity, but do we have bugs that predict mild/asymptomatic?
probably, could discuss here • Discuss potential mechanisms for
interesting OTUs - Nick

Finally, our pilot experiments testing the efficiency of infection with
different \emph{C. difficile} strains

Discuss different strain results

• Future work/conclusion section/sentence

\subsubsection{Materials and Methods}\label{materials-and-methods}

\textbf{Mice} The experiments in this study used 6-8 week-old male and
female gnotobiotic C57/BL6 mice. The mice were born and maintainted in a
completely sterile environment in the Germ Free Mouse Facility at the
University of Michigan. The University Committee on Use and Care of
Animals approved all protocols and experiments conducted in this study.

\textbf{Human samples} Donor stool samples were chosen from a sample
bank collected from patients from October 2010 to November 2012 at the
University of Michigan. Sample collection methods and study was approved
by the Institutional Review Board at the University of Michigan. Stool
samples from University of Michigan Health System patients were
submitted to the microbiology lab for \emph{C. difficile} testing.
\emph{C. difficile} presence was measured using WHATEVER- Cdiff quik
check? toxin AB?. Samples from these patients were stored in Cary-Blair
medium at -80C. All patients consented to participation in the study.
Donor stool for this study was selected from one patient that tested
positive for \emph{C. difficile} infection and five patients that
presented with diarrhea and tested negative for \emph{C. difficile}.
Healthy donor stool was collected from adult volunteers living in the
Ann Arbor, MI area. 100ul of stool slurry was gavaged into each mouse.

\textbf{\emph{Clostridium difficile} and sample processing} For all
experiments, 100 spores of \emph{Clostridium difficile} were used to
orally infect mice as described previously (cite alyx). The \emph{C.
difficile} strains used were acquired from spore stocks of (WHERE) and
comprised of strains 431, 395, 299 and 468 (cite?). After gavage, the
remaining inoculum was diluted and plated to confirm infection dose.
Bacteria were cultured anaerobically on TCCFA plates and incubated at
37C. During infection, two fresh stool pellets were collected from each
mouse daily. One pellet was resuspended in anaerobic phosphate-buffered
saline and diluted and plated on TCCFA plates to enumerate \emph{C.
difficile} CFU. The other pellet was frozen at -80C until the end of the
experiment.

\textbf{DNA isolation and sequencing} Bacterial DNA was isolated from
banked donor stool and each mouse stool pellet using the MO-BIO
PowerSoil DNA isolation kit. This DNA was used as template for

• Bacteria/plating • Sequencing • Data analysis, code availability •
Machine learning models

\subsection{Acknowledgments}\label{acknowledgments}

Lab, sequencing core, Jhansi

\newpage

\subsubsection{Figure Legends}\label{figure-legends}

\textbf{Figure 1. Germ-free mice inoculated with human feces as a model
for \emph{C. difficile infection.}} A) Experimental design. Stool from
16 healthy, diarrheal and CDI patients were independently inoculated
into 3-4 germ-free mice by oral gavage. 14 days later mice were orally
infected with 100 spores of C. difficile strain 431. Weight and stool
CFU were monitored for up to 10 days post infection. B) NMDS ordination
of donor stool communities prior to inoculating mice. Each point
represents one donor and are colored by clinical diagnosis. C) NDMS
ordination of the stool communities on day 0. Each symbol represents one
mouse and is colored by donor. Circles represent mice that experienced
mild disease and triangles represent those that suffered severe disease.

\textbf{Figure 2. \emph{C. difficile} infection results in mild or
severe disease.} A) \emph{C. difficile CFU} was enumerated by plating of
mouse stool pellets daily. Each point represents a mouse and the lines
represent the median CFU in each group. Error bars are interquartile
ranges. Red lines and points correspond to mice that succumbed to severe
disease whereas black lines and points correspond to mice that had mild
or no disease. B) Mouse weights were recorded and daily percent weight
loss calculated for each mouse. Data is presented as the median of each
group and interquartile ranges. Mice that succumbed to severe infection
typically lost a significant amount of weight by day 1 or 2 post
infection. Red lines correspond to severely ill mice, black to mice with
mild disease.

\textbf{Figure 3. Random Forest prediction of \emph{C. difficile}
colonization level.} A) OTUs above 1\% relative abundance on day 0 were
used to predict median log\textsubscript{10} CFU of \emph{C. difficile}
after colonization. OTUs were chosen such that they were not predictive
of cage or donor. Each point is a mouse colored by cage. B) Partial
dependency plots of the top six predictive OTUs. Line displats the
partial dependence of log\textsubscript{10} CFU on the relative
abundance of each predictive OUT. Each median log\textsubscript{10} CFU
is plotted against its relative abundance for each predictive OTU.

\textbf{Figure 4. Random Forest prediction of CDI severity.} OTUs above
1\% relative abundance on day 0 were used to predict disease severity.
OTUs were chosen such that they were not predictive of cage or donor.
Predictive classification tested via 10-fold (gray), leave-one-cage-out
(purple dashed) or leave-one-mouse-out (blue dashed) models are
displayed in (A). B) Partial dependency plots of most predictive OTUs.
Line displays the partial dependence of log\textsubscript{10} CFU on OTU
relative abundance. Points are the OTU relative abundance of each mouse
colored by outcome (red, severe, black, mild).

\textbf{Figure 5. Infection of mice with different \emph{C. difficile
strains}.} 3 strains of \emph{C. difficile} were used to infect mice
colonized with susceptible (DA00578) or resistant (DA00369, DA00430)
human donor stool. A) \emph{C. difficile} stool CFU was enumerated over
10 days. B) Percent weight loss was calculated each day for each mouse.
In both plots, each mouse is a point and lines represent the mean of
each cage.

\paragraph{Supplement}\label{supplement}

\textbf{Table S1: Mouse day 0 communities by donor genera (avg + stdev
of cage)}

\newpage

\subsection{References}\label{references}

\end{document}
